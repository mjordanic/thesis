
\newpage
\phantomsection
\addcontentsline{toc}{chapter}{Abstract}

\begin{abstract}

Extraction of neuromuscular information is an important and extensively researched issue in biomedical engineering. Information on muscle control can be used in numerous human-machine interfaces and control applications, including rehabilitation engineering, e.g., prosthetics, exoskeletons and rehabilitation robots. 

Neuromuscular information can be extracted at the brain level, peripheral nerves, or muscles. Among these options, muscle interface is the only viable way of information extraction in everyday life. Although brain and nerve recordings are promising, they usually require invasive measurement and achieve relatively low extraction speed which prevents real time control. Even though in electromyographic (EMG) recordings information is not obtained directly from neural cells, it contains similar information as nerve recording. Information contained in action potential of the innervated muscle fibers (MUAP) is equivalent to the information contained in the action potential of corresponding motor neurons. Moreover, muscles contain multiple motor units that activate simultaneously so their electrical activity sums on the surface of the skin, resulting in a relatively high amplitude compared to the other bioelectrical signals. Therefore, due to the richness of neural information, noninvasiveness and high signal-to-noise ratio, the surface EMG is extensively used for man–machine interfacing, especially in commercial/clinical upper-limb prosthetic control.

Motivation and merit of this thesis lies in the fact that information associated with muscular pattern during exercises can be very useful in different applications such as monitoring patients’ control strategies during recovery, personalizing rehabilitation processes to increase their effectiveness or to provide information to be used for control of external devices (EMG based control of prosthesis or exoskeletons).

Within this doctorate a pattern recognition approach was used to assess neuromuscular information and to identify subjects' intended motion based on multichannel surface electromyographic recordings. Research was focused on control strategies of upper-limb, both in normal subjects and in patients with impaired mobility caused by incomplete spinal cord injury. Methods which are proposed can be used for the design and monitoring of rehabilitation therapies intended for patients with neuromuscular impairment, as well for the control of external devices like rehabilitation robots, exoskeletons, prostheses and even virtual games. However, that is in the domain of future applications and is not the scope of the thesis.

\end{abstract}