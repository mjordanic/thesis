\newpage

\phantomsection
\addcontentsline{toc}{chapter}{Resumen}

\begin{resumen}

La magnetoencefalografía (MEG) es una técnica no invasiva de adquisición de señales cerebrales que proporciona una excelente resolución temporal y una cobertura total de la cabeza, permitiendo el mapeo espacial de las fuentes cerebrales. Estas características hacen del MEG una técnica apropiada para localizar la zona epileptogénica (EZ) en la evaluación preoperatoria de la epilepsia refractaria. 

La evaluación prequirúrgica con MEG puede orientar la colocación del EEG intracraneal (iEEG), el actual modelo de referencia en la práctica clínica, e incluso suministrar información suficiente para una intervención quirúrgica sin registros invasivos; reduciendo la invasividad, la incomodidad y el costo del diagnóstico de la epilepsia prequirúrgica. Sin embargo, las señales MEG tienen baja relación señal ruido en comparación con el iEEG pudiendo imposibilitar la detección de descargas epileptiformes interictales (IEDs) y oscilaciones de alta frecuencia (HFOs), dos importantes biomarcadores utilizados en la evaluación preoperatoria de la epilepsia.

En esta tesis, la reducción de dos tipos de interferencia está dirigida a mejorar la relación señal-ruido de la señal MEG: los artefactos metálicos que enmascaran la actividad de las IEDs; y el ruido de alta frecuencia, que enmascara la actividad de las HFOs. Debido al gran número de canales MEG y la larga duración de los registros, tanto reducir el ruido como seleccionar los biomarcadores manualmente es una tarea que consume mucho tiempo. Los algoritmos presentados en esta tesis aportan soluciones automáticas dirigidas a la reducción de interferencias y la detección de HFOs. 

En primer lugar, se presenta y valida un nuevo algoritmo automático basado en BSS para reducir interferencias metálicas mediante señales simuladas y reales. Se prueban tres métodos: AMUSE, una técnica BSS de segundo orden; y INFOMAX y FastICA, basados en estadísticos de orden superior. El algoritmo de detección automático utiliza las características conocidas de la señal producida por la interferencia metálica. Los resultados indican que AMUSE recupera mejor la actividad cerebral y permite una eliminación efectiva de componentes artefactuales.

Posteriormente, se evalúa la influencia del filtrado de artefactos metálicos en la localización de IEDs en pacientes con epilepsia focal refractaria. Se realiza una comparación entre las posiciones resultantes de dipolos de corriente equivalentes (ECDs) producidos por IEDs: sin eliminar interferencias metálicas, rechazando solamente canales con elevados artefactos metálicos y, por último, después de una reducción utilizando el algoritmo BSS desarrollado. Los resultados muestran que se logra una reducción significativa en la dispersión utilizando el procedimiento de reducción basado en BSS, lo que produce ubicaciones factibles de los dipolos en contraste con los otros enfoques.

En segundo lugar, se desarrolla un algoritmo para la detección automática ripples epilépticos en MEG utilizando sensores virtuales basados en la técnica de beamformer. La detección de ripples se realiza mediante un enfoque en dos etapas. Primero, se determina el área de interés usando beamformer. Posteriormente, se realiza la detección automática de ripples utilizando las características en tiempo-frecuencia. El rendimiento del algoritmo se evalúa utilizando registros iEEG simultáneos.

Los nuevos enfoques desarrollados en esta tesis permiten una detección no invasiva mejor de los biomarcadores interictales, que pueden ayudar a delimitar la zona epileptogénica y guiar la colocación de electrodos intracraneales, o incluso determinar estas áreas sin este tipo de registros. Como consecuencia de esta mejora en la detección, y dado que los biomarcadores interictales son mucho más frecuentes y fáciles de registrar que los episodios ictales, la evaluación prequirúrgica puede ser más cómoda y menos costosa para el paciente. 

\end{resumen}