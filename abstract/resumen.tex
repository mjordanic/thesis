\newpage

\phantomsection
\addcontentsline{toc}{chapter}{Resumen}

\begin{resumen}

La extracción de información neuromuscular es una problemática importante y extensivamente investigada en el campo de la ingeniería biomédica. La información sobre el control muscular puede ser utilizada en numerosas interfaces hombre-máquina y en aplicaciones de control, en las cuales se incluye la ingeniería de rehabilitación, que abarca la utilización de prótesis, exoesqueletos y robots para la rehabilitación. 

La información neuromuscular puede ser extraída a nivel de cerebro, nervios periféricos, o músculos. Entre estas opciones, la interfaz muscular es la única forma viable de extraer información durante la vida diaria. A pesar de que los registros de señales cerebrales y nerviosas son prometedores, normalmente se necesitan medidas invasivas y su tiempo de extracción es relativamente bajo, previniendo el control en tiempo real. Sin embargo, la información neuromuscular puede ser inferida registrando la actividad eléctrica generada por la contracción del músculo (electromiografía, EMG). A pesar de que en los registros EMG la información no se obtiene directamente por las células neurales, ésta contiene información similar a la obtenida registrando los nervios. Dado el hecho que las motoneuronas inducen potenciales de acción de las fibras musculares, la información extraída por el EMG es equivalente a la información extraída por las correspondientes neuronas motoras. Por otra parte, los músculos contienen unidades motoras múltiples que se activan simultáneamente, de modo que su actividad eléctrica se suma en la superficie de la piel, lo que da como resultado una señal con una amplitud relativamente más elevada comparada a otras señales bioeléctricas. Por lo tanto, debido a la riqueza de información neural, la no invasividad y la elevada relación señal ruido, el EMG superficial se utiliza ampliamente en interfaces hombre-máquina, especialmente en sectores clínicos y comerciales en el control prostético de la extremidad superior. 

La motivación y el mérito de esta tesis reside en el hecho de que la información asociada con los patrones musculares durante diferentes ejercicios puede ser muy útil en diversas aplicaciones tales como la monitorización de las estrategias de control de pacientes durante las terapias de recuperación, la personalización de los procesos de rehabilitación para aumentar su efectividad o para proveer información que pueda ser utilizada para el control de dispositivos externos (Control de prótesis o exoesqueletos basado en EMG). 

Dentro de esta tesis doctoral se utilizó un enfoque de reconocimiento de patrones para evaluar la información neuromuscular y para identificar la intención de movimiento de los sujetos basado en registros multicanal de EMG superficial. La investigación se centró en las estrategias de control de la extremidad superior, tanto en sujetos sanos como en pacientes con movilidad reducida causada por una lesión incompleta de la médula espinal. Los métodos propuestos pueden ser utilizados para el diseño y monitorización de terapias de rehabilitación para pacientes con deterioro neuromuscular, así como para el control de dispositivos externos tales como robots para la rehabilitación, exoesqueletos, prótesis e incluso videojuegos. 





\end{resumen}