\cleardoublepage

\addcontentsline{toc}{chapter}{Acknowledgements}

\begin{acknowledgements}

\narrowlinespacing
\begin{myquote}
\begin{flushright}
%Soyons reconnaissants aux personnes qui nous donnent du bonheur, elles sont les charmants jardiniers par qui nos âmes sont fleuries.\\
\textit{Let us be grateful to people who make us happy,\\they are the charming gardeners who make our souls blossom.} \\-- Marcel Proust
\end{flushright}
\end{myquote}
\normallinespacing

Writing this section is an act that demands more social responsibility than writing the rest of the thesis. The interesting fact is that most “readers” will pay great care only to these paragraphs and will only rapidly turn the other 150 pages by a firm grip of the thumb and the index finger, the so called \emph{key grip}.

They are the people I have met through life, whose friendship influenced me and my personality. To them I owe gratitude for who I am (or not!). They all mean a lot to me, whether we are no longer in contact, we are in contact, or we exchange faithfully four e-mails per year. They all deserve to be mentioned by name, but that would require appendix D. I must, however, mention them in a string of characters, in no particular order: T, A, Š, S, M, M, N, I, M, M, M, K, K, M, D, Dj, B, M, J, E, 1., 2., P, I, S, T, A, K, A, T, L, S, Dr. M, K, Z, V, E, A, I, R, N, D, C, V, L.

These readers are also my family, to whom I will always stand in debt. To my wonderful parents, a dangerous collision of empathy and feelings from one side, and logic and stubbornness from the other, who always provided me with more than I needed and guided me with their wisdom through the important choices in life, which I would only occasionally choose. This required patience! Here I also have to thank Mirta, my sister, with whom I grew up with. She always tolerated my misbehavior, and we share a lot of precious memories. Snow shoveling and other winter sports like eating snow are my favorites. Especially because she had to eat most of it...

I also need to express my gratitude to Albert, the man with a green thumb, a daydreamer who accepted me in his home remotely and in good faith, and greatly facilitated my acclimatization to Barcelona. His ability to produce chaos is beyond anything I encountered with, but without him I would have been lost. Luckily, I gained an ally when Pau joined me in the battle against this bohemian mess. A true companion in dust, tools, and spoiled food.

My Catalan friends who are not Catalan at all! Ivan and Zvonko with whom I shared so many \emph{glasses-of-water} all along the city, mountain experiences, long drives, swims, sweatings, and early morning yoga classes. I particularly enjoyed our discussions on physics, electromagnetism, and linear algebra. These discussions would always start with a disagreement on the particular question or phenomena, but very quickly would bring us to the elementary mathematics, where they would shortly come to an end because of a lack of knowledge and arguments. Number $\pi$, for example. Where is it coming from? It's in every important equation... why? Where is it coming from?!!

On the other hand, there are people who might find interesting what I wrote here and who might even go further and read other chapters. My colleagues from whom I have learned so much. Miguel Angel, who showed me trust from the beginning and guided me through this research with his knowledge and experience, for which I am grateful. I admire his ability to look at the problem from a different perspective and come up with a solution where there isn't any. He also has a unique and intriguing ability to postpone tasks. Mónica, who introduced me to the field, answered thousands of questions, and mentored me patiently through these years. Joan Francesc, always existing reliably in the background, giving support at the crucial moments. Hamid, a kind fellow with whom I didn't spend as much time as I wanted to. The volume of his knowledge amazes me. I also have to thank the rest of the group. To Carolina, Leidy, and Alejandro for sharing the burden and despair of a doctoral study. And to Sergio for being the faithful companion on Friday meetings. BIOART group is a lovely working environment which I grew fond of.

Finally, I have to thank to my wonderful companion Kate for standing beside me and sharing her life with me, in the beginning over Skype, and towards the end in the sobreatico of Carrer de Roger. For all the travels, adventures, ups, and downs -- everything with a smile and goofy silliness. I am grateful to have a soulmate with such a high compatibility, and a true friend. Her life energy and joy inspire me and sometimes it can be difficult to keep up with her. And I hope it always will be...

That's it! The corny cliché part is over! Everybody has been thanked to. But for the end I would also like to thank to myself. Without help I wouldn't have gone far, but without daring I wouldn't have even started. Coming to Barcelona was a lifelong experience that changed and broadened my perspective on many things. As Hunter S. Thompson wrote: “\textit{Maybe it meant something. Maybe not, in the long run, but no explanation, no mix of words or music or memories can touch that sense of knowing that you were there and alive in that corner of time and the world. Whatever it meant.}”

\end{acknowledgements}