\cleardoublepage

\addcontentsline{toc}{chapter}{Acknowledgements}

\begin{acknowledgements}
\textit{This section is written in Spanish, my mother tongue.}

Hace cinco años inicié esta tesis doctoral en la que tuve la oportunidad de trabajar con el procesado de señales aplicado a las señales cerebrales. Para mi, es la cohesión de dos temáticas que me resultan apasionantes y me siento afortunada de haberme podido dedicar a ellas exclusivamente durante estos años. Fueron 5 años de cambios y de crecimiento a nivel académico y personal que no hubiesen sido posibles sin el apoyo y la ayuda de todas las personas a las que les dedico las siguientes lineas.  

Mi primer agradecimiento es para todos aquellos profesores de los que guardo un buen recuerdo que, gracias a su pasión por la docencia, me han enseñado los fundamentos para llevar a cabo esta tesis doctoral. En especial, y dentro de esta última etapa, a Miguel Angel, por brindarme la oportunidad de embarcarme en este proyecto, por su asesoramiento, su apoyo y sus ideas; a Joan Francesc, por aconsejarme, guiarme, escucharme y en general por su gran compañerismo y amistad; y a Sergio, que ha ejercido de tutor también, por su energía, visión crítica y su gran ayuda durante estos años.

Agradezco también al Dr. Russi, por su asesoramiento en los temas de epilepsia y a Rafal por su apoyo en los temas de MEG, asi como por su colaboración y aportaciones durante estos años. Al profesor Sylvain Baillet y a todo el grupo de MEG del Montreal Neurological Institute por su cálida acogida durante la estancia y por todo lo que aprendí en ella. A Leidy y Mislav, mis compañeros de laboratorio, por la ayuda, el apoyo y la buena compañia; asi como a todas las personas que han pasado por el lab, por la uni y por el grupo de las cuales guardo un gran recuerdo.

Agradezco a mi madre, Silvia, por enseñarme los valores de la perseverancia y fortaleza; a mi padre, Ricardo, por enseñarme que nunca hay que dejar de aprender; a los dos por el apoyo incondicional y el cariño, por ser dos grandes personas que me inspiran y me sirven como modelo. A mis cuatro hermanos, Fede, Santi, Pablo y Johnny por todas las risas, los debates, las charlas de microchips, los momentos buenos y malos, pero sobre todo por ser excelentes personas a las que admiro y de las que aprendo cada día. A mis sobrinos Marcos, Lucia y Gabriel, por ser las tres personas mas eficaces para desconectar de la tesis, por la alegria, dulzura, ternura y locura. 

Agradezco a Judith, Lara, Laura, Magui y Marcel·la por todos estos años de amistad, por aguantar mis agobios y malhumores, por ser unas perfectas compañeras de cervezas, de planes, carnavales, esquiadas, aventuras, planificaciones de cumples y bodas, viajes y sobretodo por no ser nada convencionales. A Skamot, en especial a Clara, Desi y Anna, por toda la energia y por todo lo que he aprendido de ellas y junto a ellas. A Xavi y Maria por compartir frustraciones doctorandiles y enseñarme que hay luz al final del túnel. A todos los amigos y familia ampliada que no he mencionado especificamente de Vilanova, el esplai, la uni, el master, bdigital, Barcelona, Argentina y los que están desperdigados por otras partes del mundo.

Por último agradezco a Toni, por ser un pilar durante la tesis. Por toda su ayuda y asesoramiento informatico, sus cuestionamientos y preguntas que han ayudado a mejorar el trabajo hecho. Por todo lo que compartimos, por nunca rendirse en los debates de ideas, por los viajes en el ibizita para desconectar de las tesis, por Canada y por Oxford, por las crisis doctorales, que por suerte nos venian a destiempo, pero sobretodo por ser mi compañero, por escucharme y cuidarme. 

\end{acknowledgements}