\chapter{Conclusion}
\label{ch:conclusions}

%\narrowlinespacing
%\begin{myquote}
%\begin{flushright}
%\textit{The only true wisdom is in knowing you know nothing.} \\-- Socrates
%\end{flushright}
%\end{myquote}
%\normallinespacing

\narrowlinespacing
\begin{myquote}
\begin{flushright}
\textit{There is nothing insignificant in the world. \\It all depends on the point of view.} \\-- Johann Wolfgang von Goethe
\end{flushright}
\end{myquote}
\normallinespacing

\section{Summary}

Task identification and movement estimation based on the EMG are popular topics in biomedical engineering involving different areas in machine learning and, particularly, in pattern recognition with many possible applications in assistive and rehabilitation devices. The emergence of the high-density EMG (HD-EMG) opened new possibilities for extracting neural information and it has been reported that spatial distribution of HD-EMG intensity is a valuable feature in the identification of isometric tasks.

This doctoral thesis investigates further the spatial muscle co-activation patterns of myoelectric activity extracted from the HD-EMG activation maps. HD-EMG was measured on five muscles of the forearm and the upper-arm in a monopolar configuration during isometric forearm tasks. Measurements were performed on a group of healthy subjects and on a group of patients with incomplete spinal cord injury.

In chapters 3 and 4, co-activation patterns of patients with incomplete spinal cord injury were analyzed by means of pattern-recognition-based identification of task and effort level. Both intensity related features, and spatial features were analyzed. In chapter 3, co-activation patterns were analyzed for each patient individually, whereas in chapter 4, co-activation patterns were analyzed within the group of patients. In spite of the great diversity between different patients and their levels and types of injury, similarities between activation patterns were found not only in the intensity of the myoelectric signal, but also in the spatial distribution expressed as center of gravity.

In the chapter 5, a novel feature for task identification was proposed. The feature is based on spatial distribution of the myoelectric activity recorded by the HD-EMG and analyzed by the mean shift algorithm. This new feature was evaluated in identification of task and identification of task and effort level in healthy subjects. The evaluation was performed for each subject individually. 


\section{Discussion and conclusions}

In chapter 3, muscle co-activation patterns were analyzed during four isometric tasks and three effort levels in patients with incomplete spinal cord injury. Intensity-based activation maps were calculated for each muscle and different features were extracted: the average intensity of an HD-EMG map, and the center of gravity of an HD-EMG map. Using the extracted feature sets, a successful patient-specific task identification method was designed. It is capable of estimating with high sensitivity and precision not only the motor task ($S = 97\%$, $P = 97\%$), but also the force ($S = 92\%$, $P = 93\%$). This implies that patients with incomplete spinal cord injury have a repeatable co-activation muscular patterns not only in intensity, but also in spatial distribution of intensity over the muscle surface. Moreover, the results lead to the conclusion that the spatial distribution of the myoelectric activity has a significant and discriminative power in classification. Furthermore, adding information on the spatial distribution of myoelectric intensity improves not only the identification result, but also the resilience to fatigue ($S = 84\%$, $P = 87\%$) and time effect ($S = 94\%$, $P = 95\%$).

Furthermore, in chapter 4 it was discovered that repeatable patterns in intensity and spatial distribution besides existing for each patient individually, also exist for the entire group of patients. To demonstrate the existence of distinguishable group-specific patterns in HD-EMG, the identification of different tasks was performed, where the classifier was trained and tested using the samples of all patients instead of focusing on individuals, i.e., a group-specific classifier was designed. The existence of the similar patterns in individuals is an interesting result because there should be a high level of variability between patients due to the nature of the injury. Co-activation patterns were found not only between different tasks, where task identification reaches but also between different effort levels. Group-specific identification of motion intention in patients with a neuromuscular impairment could potentially improve the translation of pattern recognition techniques to clinical practice. Also, results show that the similarity is greater between patients with a similar level of lesion (vertebra at which spinal cord injury occurred). This could also have an interesting implication in the translation to the clinical practice because patients with the similar level of injury could be able to use the same assisitive/rehabilitation devices with greater ease, avoiding or at least facilitating the calibration process that is intrinsically needed when the training of the system is subject specific. This is an important result because one of the reasons preventing the broad use of pattern recognition approaches for the control of assistive devices is precisely the calibration process that is time-consuming and burdensome for the user.

Finally, in chapter 5, a novel feature for the identification of task and effort level was designed. It is based on the locations of the local maxima of the probability density function of HD-EMG activation maps, and is obtained using the mean shift algorithm. The feature was tested on the population of healthy subjects in a subject-specific approach, that is, the classifier was trained and tested for each subject individually. The feature yields higher identification indices compared to the more classical features (sensitivity and precision almost 100\% in task identification and 97\% in identification of task and effort level), especially in the task identification at very low effort level. By analyzing the influence of fatigue and other time-dependent changes (e.g. drying of conductive gel) on identification, the novel feature had a very good performance. Since the goal of this study was to analyze different feature sets rather than classification methods, LDA was utilized given that this method is the most commonly used, and is generally recommended for myoelectric interfaces \citep{Hakonen2015}.

Density function from which the modes were extracted represents the RMS activation maps of the HD-EMG. Although the feature proved to be useful, by calculating the RMS value, the complexity of the underlying system is reduced, but the information is also partially lost. Therefore, the modes, or other statistical measures of the raw HD-EMG, i.e. joint distribution of instantaneous EMG amplitude over the electrode array, could also be a useful feature in the identification of the motion intention. Furthermore, in the literature, features are often calculated for each channel separately and then the optimal set of channels is selected prior to the classification using the, e.g., sequential method \citep{Hargrove2009, Li2017}, selection based on common spatial patterns \citep{Geng2014}, or based on the independent component analysis clustering \citep{Naik2016}. Modes of the HD-EMG density function could be correlated with the channels with discriminative information and could be a useful tool in the channel selection.

Moreover, the mean shift algorithm can be used for clustering and, since it was shown that the algorithm is most effective in the low-dimensional data, image segmentation is one of its most successful applications \citep{Comaniciu2002}. A mode of the density estimate, or in this case, a channel selected by the mean shift algorithm, can be considered as a cluster representative \citep{Hennig2015}, related to the possible image segments, where spatial (pixel locations) and range features (the intensity of the grayscale value) are considered. The advantage of the mean shift is that it can be used for clustering non-convex shapes, albeit, it could segment complex non-convex regions in the activation maps. Since segmentation of the muscle activation map can improve the neuromuscular activity estimation \citep{Vieira2010}, this could be a reason why mean shift features improved the performance of the detection system compared to the previously published attributes. In addition, the algorithm only requires setting of one parameter, the bandwidth ($h$) and, unlike in the similar methods, it is not necessary to define the number of expected clusters. This is a big advantage because it does not require a priori knowledge on the number of clusters.

The proposed motor task identification method based on spatial information of myoelectric distribution could contribute to human-machine interface technology. There are many possible applications for this type of technology, for example computer games, exoskeletons, automatic wheelchairs, rehabilitation robots, prostheses, etc. Nowadays, the field of the brain-computer interface (BCI) technology is advancing very fast with high investments of leading global corporations. However, the non-invasive BCI is still an open problem with a low output rate, which can be greatly improved by using EMG-based identification of motor intention. For example, Müller-Putz et al. suggest non-invasive hybrid brain-computer interfaces (hybrid BCI) designed as EEG-based system, supplemented with other biological and mechanical signals  \citep{Muller-Putz2015}. Joining EEG and EMG recordings in the identification of the task intention significantly improves the accuracy of the individual EEG or EMG system. The EMG usually has a higher SNR ratio than the EEG and it is widely used in the identification of the motion intention, however, it is prone to malfunction due to, e.g. fatigue. When fatigue occurs, the supplemented EEG input keeps the identification stable, and increases the robustness of the system. Thus, advances in obtaining methods more robust to fatigue or time effect are very interesting.

Some patients with neuromuscular impairment can weakly activate their muscles, but insufficiently to generate a movement. In these patients, as well as in patients that can generate only weak movements, HD-EMG maps can still be generated and used in identification of motion intention, as demonstrated in this study. This approach could potentially supplement the existing BCI or inertial sensors based prostheses and result in a device with a better performance. For example, \citet{Rohm2013} performed a very interesting study with a single SCI patient. Their neuroprosthesis consisted of a functional electrical stimulation of the forearm and upper arm muscles, and a semiactive elbow orthosis. Using BCI and a shoulder joystick, the patient was able to perform complex hand and elbow tasks from everyday life (e.g. eating an ice cream cone). The reported performance of that study was 70\%, which was remarkable considering the fact that the patient did not have any control over the involved muscles. However, performance of similar patients could be increased by using hybrid BCI if myoelectric activation exists.

%As a limitation of the thesis, it should be noted that the proposed features were tested only in highly controlled conditions of isometric contractions. The experiments during non-isometric contractions should be performed in order to validate the quality of the features in dynamic and more natural movements. Also, the experiment included only four tasks related to the elbow joint. Further analysis should include a higher number of more complex tasks related to hand and shoulder. Within the scope of this thesis, advancements have been made in that direction. The experimental protocol was already designed and HD-EMG signal was recorded in healthy subjects during shoulder-related hand movements in horizontal plane, similar to the movements stimulated by rehabilitation robots during the therapy. Identification of non-isometric tasks was performed on that recordings and the results were accepted for publishing \citep{Rojas-Martinez2017} and can be found in the appendix \ref{RIAI}. Moreover, all results in this thesis were obtained during offline analysis. To evaluate practical aspects of the technique, the experiment should be repeated using online identification.


\section{Main contributions}

The original contributions provided by the compendium of publications of this thesis are:

\begin{itemize}
\item The definition of a pattern-recognition algorithm for task and force identification in the group of patients with incomplete spinal cord injury. The method was based on the combination of intensity and spatial distribution of intensity of the myoelectric signal, namely center of gravity. The algorithm was validated in terms of robustness during slow time dependent changes, such as fatigue and drying of conductive gel. The results prove the existence of the repeatable co-activation pattern in intensity and spatial distribution for each patient. Furthermore, the pattern exists for different tasks, but also for different effort levels.

\item The co-activation pattern in intensity and its spatial distribution of HD-EMG was identified for the group of patients with spinal cord injury. It was proved that there is a coherence between activation patterns of different patients after the injury, both regarding task and force. This coherence can be observed in the intensity of the HD-EMG, but also in the spatial distribution of intensity. Furthermore, a greater similarity was found within the group of patients with a similar level of injury. This result implies the possibility of building assistive/rehabilitation devices for the group of patients with a significantly lower training time.

\item Definition of a novel statistical spatial feature derived from the HD-EMG. It was used for the identification of task and effort level in a group of healthy subjects. This feature is based on the probability density function of the HD-EMG activation map and is suitable for real time identification. It achieves particularly good identification results of contractions performed at very low effort level, as well as a high robustness to fatigue.


\end{itemize}

\section{Future Work}
\label{sec:fw}
The work developed in this thesis opens new possibilities in the field of myocontrol in rehabilitation. Some of the most interesting possibilities for future works are the following: 

\begin{description}
\item[Expand the database] \hfill \\
	In order to confirm the results presented in this thesis, the study should be repeated on a database containing more subject. Also, the experiment included only four tasks related to the elbow joint. Further analysis should include a higher number of more complex tasks related to hand and shoulder.

\item[Dynamic contractions] \hfill \\ 
	The use of spatial information of myoelectric activity is a novel method which already showed very good results in the identification of tasks, both in healthy subjects, and in patients with incomplete spinal cord injury during isometric contractions. Isometric contractions are standard to the field of work, that is, pattern recognition for control of human-machine interfaces, and are a good starting point to test new features with respect to more classical features. The recordings during isometric contractions provide measurements with more controlled conditions, i.e., minimized influences related to relative shift of recording electrodes with respect to source of the signal – muscle fiber. Therefore it is a good practice to start using new features in controlled situations in order to establish reliably and precisely the circumstances in which features are useful. However, the experiments during non-isometric contractions should be performed in order to validate the quality of the features in dynamic and more natural movements. One of these studies was already performed within the scope of the thesis and the results were published \citep{Rojas-Martinez2017} and can be found in the appendix \ref{RIAI}. Identification was performed of dynamic shoulder-related movements of the hand in the horizontal plane, similar to the movements in rehabilitation. The same feature related to the spatial distribution was used -- center of gravity. The identification has a very promising results (sensitivity and precision close to 90\%), both at high speed movements, and low speed movements. This study should be repeated on a database containing a higher number of subjects, but also number of movements should be increased. Furthermore, the dynamic task identification using the novel feature based on the mean shift algorithm, proposed in this thesis, should be tested. 
	
	
%	However, isometric contractions The  the merit of combination of intensity and spatial features for identification of specific group of tasks during isometric contractions, and extends the analysis to identification of dynamic tasks. Research is focused on control strategies of upper-limb in normal subjects. Tasks analyzed in the study are related to hand movement in horizontal plane, which mostly involves movements at shoulder joint. These movements were found important because they are commonly used in rehabilitation robots. Although this research was performed on healthy subjects, methods which are proposed could be used for the design and monitoring of rehabilitation therapies intended for patients with neuromuscular impairment. 
	
%	We are already planning and developing the new framework within which we will record non-isometric upper-limb tasks. In our opinion, real time identification of non-isometric tasks should be the final goal of the project.

\item[Generalized mean shift approach] \hfill \\
	In chapter 5, the motor task identification algorithm that uses the novel spatial feature is explained. This spatial feature is based on the modes of the probability density function of HD-EMG activation maps. Instead, the viability of features based on the  modes of the probability density function of raw HD-EMG signal should be explored. Since the information is partially lost by calculating the RMS value of the signal to obtain the activation maps, using the joint distribution of instantaneous EMG amplitude over the electrode array could provide higher identification results.
	
\item[Mean shift approach for channel selection] \hfill \\
	Geng et al. recently proposed a more advanced channel selection method based on common spatial patterns \citep{Geng2014} and Naik et al. propose channel selection based on independent component analysis \citep{Naik2016}. Modes of the HD-EMG density function, a novel feature proposed in chapter 5, could be correlated with the channels with discriminative information and could be a useful tool in channel selection.

\item[Real time application] \hfill \\
	The task identification system cannot find an application without the ability of online processing. During offline identification, subject cannot receive the feedback of his performance, i.e., task identification is done in the open control loop. When the task identification is performed in real time, the control loop is closed, i.e., user can receive feedback and adjust the contraction to optimize the task identification, but also compensate for the non-stationarities in the signal. Therefore, an appropriate recording device along with an optimized processing unit should be built. The device should be able to process the task identification in real time using an optimized firmware.
	
\item[Regression-based approach] \hfill \\
	Contrary to classification, continuous control information for each DoF separately can be achieved by regression. This inherently implies simultaneous control of multiple DoFs. Although it was already hypothesized that the regression is a more suitable technique in myocontrol applications \citep{Jiang2012}, the actual research that confirms this statement was only recently published by \citet{Hahne2017}. The features investigated in this thesis should be translated and evaluated in regression-based myocontrol. 

\item[Hybrid brain-computer interface] \hfill \\
	The fusion of EEG and EMG could further improve results of upper-limb task identification, that is, the study performed here using only HD-EMG recordings both in healthy subjects and iSCI patients. This type of study can have an impact on numerous fields of application including brain – computer interfaces (BCI). A goal could be to exploit the fusion of cerebral and neuromuscular information and to quantify the improvements when the innovative technique of HD-EMG is joined with the cerebral activity, what was recently called by the research community a \emph{hybrid BCI} \citep{Muller-Putz2015, Rohm2013}.

\item [Increase of identification fidelity] \hfill \\
	Fidelity of the identification could be increased further by using an adaptive model of a classifier that is being constantly updated throughout the exercise in order to compensate for the changes in the myoelectric signal caused by, e.g., fatigue or sweating. There are several recent publications on this subject \citep{Hahne2015, Vidovic2016, Sensinger2009}.

\item[Spatial distribution of frequency] \hfill \\
	Features extracted from frequency/scale domain proved to be very useful in the identification of motor task \citep{Oskoei2007}. In future works, it would be interesting to investigate the spatial distribution of frequency over the muscle in search of a discriminative feature.

\end{description}

\section {Publications derived from the thesis}
\subsection{Journal papers}

\begin{itemize}
\item Jordanić, M., Rojas-Martínez, M., Mañanas, M.A., Alonso, J.F., Marateb, H.R. A Novel Spatial Feature for the Identification of Motor Tasks Using High-Density Electromyography. \textit{Sensors}, 17(7): 1597, 2017, JCR 2.077, Q1 in Instruments and instrumentation (10/58)

\item Rojas-Martínez, M., Alonso, J.F., Jordanić, M., Romero, S., Mañanas, M.A. Identificación de tareas isométricas y dinámicas del miembro superior basada en EMG de alta densidad. \textit{Revista Iberoamericana de Automática e Informática Industrial}, Accepted for publication 2017, JCR 0.500, Q4 in Automation and Control Systems (57/60)

\item Jordanić, M., Rojas-Martínez, M., Mañanas, M.A., Alonso, J.F. Prediction of isometric motor tasks and effort levels based on high-density EMG in patients with incomplete spinal cord injury. \textit{Journal of Neural Engineering}, 13(4): 46002, 2016, JCR 3.465, Q1 in Biomedical Engineering (13/77)

\item Jordanić, M., Rojas-Martínez, M., Mañanas, M.A., Alonso, J.F. Spatial distribution of HD-EMG improves identification of task and force in patients with incomplete spinal cord injury. \textit{Journal of NeuroEngineering and Rehabilitation}, 13(1): 41, 2016, JCR 3.222, Q1 in Rehabilitation (3/65)
\end{itemize}


\subsection{Conference papers}

\begin{itemize}
\item Jordanić, M., Rojas-Martínez, M., Mañanas, M.A. Muscle pattern from HD-EMG applied to identification of movement intention. Summer School on Neurorehabilitation (SSNR 2015), 2015, Valencia, Spain

\item   Jordanić, M., Rojas-Martínez, M., Mañanas, M.A., Alonso, J.F. Use of frequency features of HD-EMG in identification of upper-limb motor task. \textit{Cognitive Area Networks}, 4(1): 19:23, 9. Simposio CEA de Bioingeniería: Interfaces Cerebro-Máquina y Neurotecnologías para la Asistencia y la Rehabilitación, 2017, Badalona, Spain

\item Jordanić, M., Rojas-Martínez, M., Alonso, J.F., Migliorelli, C. Mañanas, M.A. Identificación de Contracciones Isométricas de la Extremidad Superior en Pacientes con Lesión Medular Incompleta mediante Características Espectrales de la Electromiografía de Alta Densidad (HD-EMG). Jornadas de Automática (awarded as the best paper in the Bioengineering section), 2017, Gijon, Spain

	
\end{itemize}


