\chapter{Conclusion}
\label{ch:conclusions}

\section{Summary}

Magnetoencephalography is a noninvasive technique that provides excellent temporal resolution and a whole-head coverage that allows the spatial mapping of cerebral sources. These characteristics make MEG an appropriate technique to localize the epileptogenic zone (EZ) in the preoperative evaluation of pharmacoresistant epilepsy. There is great interest in using interictal biomarkers to determine the EZ because these can occur more frequently than seizures \citep{Staba2014} and reduce the discomfort of the patient during the recordings. 

Presurgical evaluation with MEG can guide the placement of invasive electrodes, the current gold standard in the clinical practice, and even supply sufficient information for a surgical intervention without invasive recordings, reducing invasiveness, discomfort, and cost of the presurgical epilepsy diagnosis \citep{Aydin2015}. However, MEG signals have low signal-to-noise ratio compared with intracranial EEG and can sometimes be contaminated by noise that mask or distort the brain activity. This may prevent the detection and the localization with interictal epileptiform discharges (IEDs) and high frequency oscillations (HFOs), two important biomarkers used in the preoperative evaluation of epilepsy. 

To achieve the main goal of this thesis, namely the development and validation of methods for noninvasive localization of interictal biomarkers with MEG, its signal-to-noise ratio must improve. The reduction of two kinds of interference was aimed: metallic artifacts that affect exclusively MEG recordings and mask the activity of IEDs; and the high-frequency noise, produced mainly by muscular interferences that mask HFO activity. Considering the large number of MEG channels and the long time of the recordings, reducing noise and marking events manually is a time-consuming task. The algorithms presented in this thesis provide automatic solutions aimed at the reduction of interferences and, the detection of HFOs. 

In chapter \ref{ch:3}, an automatic BSS-based algorithm to reduce metallic interference was developed and validated using simulated and real signals \citep{Migliorelli2015}. Three methods were evaluated: AMUSE, a second-order BSS technique; and INFOMAX and FastICA, based on high-order statistics. To objectively evaluate the effectiveness of BSS and the subsequent interference reduction, simulated signals consisting in real artifact-free data mixed with real metallic artifacts were generated. Subsequently, an automatic detection of the artifactual components was proposed, exploiting the known characteristics of metallic-related interferences. Results indicated that AMUSE performed better when recovering brain activity and allowed an effective removal of artifactual components.

In chapter \ref{ch:4} the influence of metallic artifact filtering using the previously developed algorithm was evaluated in the source localization of IEDs in patients with refractory focal epilepsy \citep{Migliorelli2016}. In this study, a comparison between the resulting positions of equivalent current dipoles (ECDs) produced by IEDs was performed: without removing metallic interference, rejecting only channels with large metallic artifacts and after BSS-based reduction. The results showed that a significant reduction on dispersion was achieved using the BSS-based reduction procedure, yielding feasible locations of ECDs in contrast to the other two approaches. 

In chapter \ref{ch:5} an algorithm for the automatic detection of epileptic ripples in MEG using beamformer-based virtual sensors was developed \citep{Migliorelli2017}. The automatic detection of ripples was performed using a two-stage approach. In the first step, beamforming was applied to the whole head to determine a region of interest where more high frequency activity was taking place. In the second step, the automatic detection of ripples was performed using the time-frequency characteristics of these oscillations. The performance of the algorithm was evaluated using intracranial EEG recordings as gold standard. Furthermore, the number of events that were detected inside the region of interest were significantly higher than the number of events found outside. 

The main conclusions of the different studies and its relationship with the objective of the thesis are summarized in the following section.

\section{Main conclusions}

%JNER
Nine subjects with iSCI performed four isometric forearm tasks (flexion, extension, supination, and pronation) at three levels of effort (10\% MVC, 30\% MVC, and 50\% MVC). High density EMG was measured on five muscles of forearm and upper arm in monopolar configuration. Intensity maps were calculated for each muscle and three different feature sets were extracted: the average intensity of an HD-EMG map (I), the intensity and center of gravity of an HD-EMG maps (I + CG), and the intensity of a single differential channel (Diff) (gold standard). Using the extracted feature sets and LDA-based classification, both task and effort level were identified, and the influence of fatigue and other time-dependent changes (e.g. drying of conductive gel) on identification was evaluated. Since the goal of this study was to analyze different feature sets rather than classification methods, LDA was utilized given that this method is the most commonly used, and is generally recommended for myoelectric interfaces \citep{Hakonen2015}. Although it assumes normal distribution of patterns in each class, it has proven to have good performance even when the normality assumption does not hold \citep{Grouven1996}.

When identification using the different features was tested on signals recorded in short time intervals, the combination of I + CG outperformed the other feature sets. The results show that a muscular co-activation pattern exists not only for the task intention ($Acc = 98.7\%$; $S = 96.8\%$; $P = 97.0\%$; $SP = 99.2\%$), but also for the force intention ($Acc = 98.8\%$; $S = 92.5\%$; $P = 93.2\%$; $SP = 99.4\%$).

Although the identification based on the features Diff has slightly better performance in average than the identification based on the features I, a repeated measures ANOVA showed that there is no significant difference in their distributions. Moreover, a small displacement in the position of bipolar electrodes can have a great effect on signal intensity, as well as on spectral content. Consequently, if using Diff as features in classification, a small displacement can have a high influence on the identification performance. This effect does not exist in feature I, making it more robust to small changes in the position of the electrodes. On the other hand, the identification based on the combination of intensity and spatial features significantly outperforms both of them. This result was obtained both for identification of tasks and identification of tasks and effort levels. Furthermore, it has been shown that the classifier based on I + CG discriminates between types of tasks at low levels of effort (10\% MVC) significantly better than the classifiers based on the other feature sets (Figure \ref{fig:1-5}).

The impedance between electrodes and skin changes during time on account of several causes, e.g., drying of conductive gel and sweating. Consequently, the identification performance deteriorates as the time between the training of the classifier and the identification increases. When the identification is performed long after the training of classifier, the results show that the identification based on I + CG performs just slightly better than the identification based on I features, while the identification based on Diff features is much worse ($S_{I+CG} = 94\%$, $P_{I+CG} = 95\%$; $S_I = 93\%$, $P_I = 94\%$; $S_{Diff} = 83\%$, $P_{Diff} = 83\%$). Although it may seem that, in average, spatial features do not improve the classification with respect to using only the intensity of an HD-EMG map, it is important to outline that these results were obtained on contractions of high levels of effort (50\% MVC), where performances were similar even when contractions were recorded at the same time (see \ref{fig:1-5}).

Muscle fatigue also affects the recorded EMG signal both in the time and spectral domains and therefore the identification performance deteriorates with fatigue. The results of this work show that the classifier based on intensity and spatial features is less sensitive to fatigue than classifiers based on the other feature sets. The proposed classifier shows a very good performance in task identification even at the final stage of fatigue ($Acc = 91.3\%$, $S = 84.3\%$, $P = 87.0\%$, $SP = 93.5\%$).

The proposed method could significantly improve the human-machine interface technology and can be used in numerous applications: computer games, exoskeletons, automatic wheelchairs, rehabilitation robots, prostheses, etc. As suggested by Müller-Putz et al. \citep{Muller-Putz2015}, non-invasive hybrid brain-computer interfaces (BCI) can be designed as EEG-based BCI supplemented with other biological and mechanical signals. For example, they reported significantly higher identification results for motion intention when using a hybrid BCI system composed of EEG and EMG sensory systems than when using only one of them. EMG usually has higher SNR ratio than EEG and it is widely used in the identification of the motion intention, however, it is prone to malfunction due to fatigue. When fatigue occurs, the supplemented EEG input keeps the identification stable, and increases the robustness of the system. Thus, advances in obtaining methods more robust to fatigue or time effect are very interesting.

Some patients with neuromuscular impairment can weakly activate their muscles, but insufficiently to generate a movement. In these patients, as well as in patients that can generate only weak movements, HD-EMG maps can be generated and used in identification of motion intention, as demonstrated in this study. This approach could supplement the existing BCI or inertial sensors based prostheses and result in a device with a better performance. For example, Rohm et al. \citep{Rohm2013} performed a very interesting study with a single SCI patient. Their neuroprosthesis consisted of a functional electrical stimulation of the forearm and upper arm muscles, and a semiactive elbow orthosis. Using BCI and a shoulder joystick, the patient was able to perform complex hand and elbow tasks from everyday life (e.g. eating an ice cream cone). The reported performance of that study was 70\%, which was remarkable considering the fact that the patient did not have any control over involved muscles. However, performance of similar patients could be increased using hybrid BCI if myoelectric activation exists.

Furthermore, compared to inertial signals, which are also used as input to control devices, EMG has a major advantage because myoelectric activation precedes the actual movement, which can save valuable response time.

However, it should be noted that although this study represents an improvement in the identification of motion intention, additional experiments should be considered in the future. Firstly, HD-EMG recordings were carried out during controlled isometric submaximal contractions, i.e. patient’s arm was fixed and supported by a mechanical brace. Since the methodology was capable to successfully and automatically differentiate between none, very low, low and medium effort levels, we might hypothesized that the method can be useful in prediction without the support of the brace. However, more experiments without the brace and the analysis of the recorded HD-EMG signals would be necessary to confirm and quantify this hypothesis.

%\section{Conclusion}
In this study, the spatial distribution of EMG intensity was evaluated for identification of tasks and different levels of effort in patients with iSCI. Results show that the spatial activation of motor units is dependent on the type of exercise and contraction intensity, and that related features can improve identification performance.

Although results show that spatial features also enhance the robustness of the identification to time effect and fatigue, additional experiments need to be performed to test robustness to temporal dependent changes more thoroughly and to determine when the classifier fails by further tests done on fatigue.

The center of gravity was used as a figure of merit to describe the spatial distribution. Although it shows a significant improvement in classification, by definition it is insensitive to fine changes in the distribution of muscle units. Therefore, in future works, more appropriate measures of spatial distribution should be analyzed in order to better describe the spatial distribution of muscle intensity. Also, additional features as those related to the frequency content could be considered to improve even more the classification performance.


%JNE
In order to demonstrate the existence of distinguishable group-specific patterns in HD-EMG, the identification of different tasks was performed. Within-group identification of motion intention at different effort levels was tested on nine patients with iSCI performing four upper limb tasks (flexion/extension of the elbow and supination/pronation of the forearm) at three different effort levels (10\%, 30\%, and 50\% MVC). 

Although a single type of a classifier would be sufficient to demonstrate the existence of different patterns, for an additional verification two types of classifiers were evaluated in the identification of motion intention: LDA and SVM. The former is a classical, simple, and computationally efficient classification method, whereas the latter is a more powerful classifier that can employ a nonlinear transform of features to improve their separability among classes. In this paper, a SVM with radial kernel was considered. Although the SVM is superior in classification performance, the LDA is commonly used in myocontrol applications because of its simplicity and performance in real-time. However, with the increasing computational power of new computer generations, SVM could become more common in these applications. 

The identification of tasks was tested using two feature sets: 1) the average intensities of HD-EMG activation maps (I) of five muscles and 2) the combination of average intensities and centers of gravity (I+CG) of the activation maps of five muscles.
On the other hand, a conjoint identification of tasks and effort levels was designed as two-step classifier, following the procedure described by Rojas et al. \citep{Rojas-Martinez2013} and tested on a healthy population. The first step comprised the identification of tasks using a combination of intensity and spatial features of all five muscles, whereas in the second step the levels of effort were identified separately for each task. The effort levels were identified using a combination of the intensity and spatial features of agonist-antagonist muscle pairs involved in the task \citep{Rojas-Martinez2013}.

HD-EMG activation maps were calculated for all exercises and compared among patients. 

Rojas-Martínez et al. \citep{RojasMartinez2012} calculated the relative standard deviation between maps within a group of healthy subjects (17.4\% in average), reporting an increase in standard deviation between maps with increasing effort levels (12.1\%, 16.6\%, and 23.6\% for 10\%, 30\%, and 50\% MVC, respectively). As expected, the dispersion between maps of iSCI patients was considerably higher (56\% in average), but the variability was similar in the case of patients with iSCI (Table 1). However, when maps were compared among patients with the same level of injury, the standard deviation between maps was greatly reduced (19\% in average). Moreover, the variability was higher for muscles of the upper-arm (biceps and triceps) than for forearm muscles. This reduction could be either due to a distinct activation, specific to the level of injury, or because during the rehabilitation process patients developed similar activation patterns. This is an important finding that has to be taken into account when training a classifier for a group of patients. Muscle activation patterns in patients differed from those of healthy subjects in \citep{Rojas-Martinez2012}: the Biceps Brachii was more active during supination than during flexion; the Pronator Teres was more active during supination and especially during flexion than during pronation. This could be because both muscles are particularly affected by the iSCI at the level of C4 \citep{Young-SCI}.

Furthermore, the results using the LDA showed much better identifications within the group of patients with a C4 level of injury than within the group of all patients. These findings could be related to a higher homogeneity among patients with the same level of injury. The combination of intensity and center of gravity performed better than only intensity features. These results showed that similar patterns exist in spite of the diverse nature of their injuries. This correlation exists not only in the average intensity of the HD-EMG activation maps, but also in the spatial distribution of EMG intensity, which justifies the choice of these intensity and spatial features for automatic identification.

Finally, a considerable improvement was observed when using the SVM instead of the LDA, reaching the following results: 1) excellent automatic task identification even in the group of all patients ($Acc=99.0\%$, $S=97.9\%$, $P=98.0\%$, and $SP=99.3\%$), 2) a good combined classification of four tasks and three effort levels also in the group of all patients ($Acc=97.5\%$, $S=85.2\%$, $P=85.3\%$, and $SP=98.7\%$) which is even better in 3) conjoint identification of four tasks and low or moderate effort levels ($Acc=98.0\%$, $S=92.0\%$, $P=92.4\%$, and $SP=98.9\%$). In spite of the previous reports suggesting the greater importance of selection of the features than the selection of the classifier, our results have shown that both have considerable impact on the identification.

Several array subsets corresponding to 3$\times$3 square grids of channels (IED = 10 mm) located at different positions were also used to evaluate the possibility of task identification using a much smaller number of electrodes. In this case, the results were considerably worse, especially when using the LDA classifier. Due to the small region covered by electrodes in each muscle, the spatial information could not be extracted and it was not possible to increase the performance as in the case of using all the electrodes.

Although this study presents an important improvement in the identification of motion intention, it is important to mention that the recordings were carried out during highly controlled isometric contractions. Therefore, even though the findings are promising, they are only a step towards final real-time applications involving free movements and multiple DoFs.

The results show that the use of a SVM-based classifier is indeed a promising approach in myocontrol-oriented pattern recognition applications. Moreover, even though a different activation pattern can be expected in subjects with neurological impairment, as in the present case, such pattern can still be associated with task and level-dependent changes in the spatial distribution of the intensity, as has been previously observed in non-injured subjects \citep{Rojas-Martinez2012}.


%\section{Conclusions}
Group-specific identification of motion intention in impaired patients has a potential to improve the translation of pattern recognition techniques to clinical practice. Unfortunately, group-specific design is a difficult topic because it assumes strong task-related and level of effort-related co-activation patterns among patients, but given the diverse nature of injuries and the high inter-patient variability, co-activation patterns are weak.

This study shows that muscular co-activation patterns in intensity and spatial distribution indeed exist. Furthermore, it shows that stronger co-activation patterns can be found between patients of the same level of injury. Whether because of the rehabilitation process or the level of injury, muscle control strategies are similar for the group of patients with an injury at C4, which makes them a more homogenous population and enables the control of universal assistive devices with higher reliability. In summary, in spite of the difficulty to identify both task and effort level in patients with iSCI, very promising results were found to provide a useful estimation of motion intention. 


%Sensors

This study showed that the combination of intensity and spatial information is useful for the extraction of neuromuscular information. The spatial information was calculated from the RMS activation maps using the mean shift algorithm. Results were evaluated using the 70\% repeated holdout method and stratified sampling as to have sufficient number of samples of each class in the sets. To prevent the type III statistical error \citep{Mosteller1948, Mohebian2017}, a repeated hold-out was used. Sensitivity and precision, as appropriate unbiased measures in analyzing imbalanced multi-class problems \citep{Jordanic2016b, Rojas-Martinez2013}, were used to quantify the identification.

IMS features achieved very good results compared to other feature sets during task identification when the task was performed at very low effort level. Moreover, the Friedman test showed no significant differences in task identification using IMS when tasks were performed at 10\% MVC, 30\% MVC, or 50\% MVC. This can be a very important quality in everyday applications where subject could not need to contract muscles at moderate effort level to complete the task. It can be a step toward more natural control where even slight contractions can be successfully identified. In fact, only activations with low level of intensity are sometimes possible in patients with neuromuscular impairments.

A high identification rate is not the only factor important in the extraction of neural information from sEMG. The system should also be robust to slow time-dependent changes such as fatigue and electrode-skin contact impedance \citep{Farina2014}. Therefore, the robustness of the proposed features was tested with respect to time and fatigue. When evaluating the time effect, no significant differences in performance were found between IMS, ICG, and I feature sets and IMS significantly outperformed TD and Diff features. However, time effect was evaluated only when test set was composed of contractions recorded at 50\% MVC and, as shown in Figure \ref{fig:3-6}, all features perform similarly for the identification of that effort level. This phenomenon was already remarked and described in \citep{Jordanic2016a} where authors noted that adding spatial features to intensity features significantly improved the identification of tasks recorded at low effort levels, whereas improvement is not significant at moderate effort levels. On the other hand, the proposed features are particularly robust in task identification during fatiguing exercises and show significantly higher identification rate when compared to other features. Further improvements in reliability of the identification during the long-term contractions and fatiguing contractions can be achieved by using adaptive identification models that are being constantly updated during the usage (e.g., \citep{Vidovic2016, Hahne2015, Sensinger2009}).

In the current work, features were extracted from the RMS activation maps of the HD-EMG. Although these features proved to be very effective, by describing the EMG signal with its RMS value, i.e., the estimator of variance, the information is partially lost. Since the gradient of the probability density function of raw EMG is a useful feature in task identification, statistical measures (e.g., modes) of the raw HD-EMG, i.e., joint distribution of instantaneous EMG amplitude over the electrode array, could provide valuable information. Moreover, in literature, features were often calculated for each channel separately and selected using the simple sequential method prior to classification \citep{Hargrove2009, Li2017}. On the other hand, Geng et al. recently proposed a more advanced channel selection method based on common spatial patterns \citep{Geng2014}. Modes of the HD-EMG density function could be correlated with the channels with discriminative information and could be a useful tool in channel selection.

Finally, the mean shift algorithm can be used for clustering and, since it was shown that the algorithm is most effective in low-dimensional data, image segmentation is one of its most successful applications \citep{Comaniciu2002}. A mode of the density estimate, or in this case, a channel selected by the mean shift algorithm, can be considered as a cluster representative \citep{Hennig2015}, related to the possible image segments, where spatial (pixel locations) and range features (the intensity of the grayscale value) are considered. The advantage of the mean shift is that it can be used for clustering non-convex shapes, albeit, it could segment complex non-convex regions in the activation maps. Since segmentation of the muscle activation map can improve the neuromuscular activity estimation \citep{Vieira2010}, this could be a reason why mean shift features improved the performance of the movement detection system compared with previously published attributes. In addition, the algorithm only requires setting one parameter, bandwidth ($h$) and, unlike in the similar methods, it is not necessary to define the number of expected clusters. This is a big advantage because it does not require a priori knowledge on the number of clusters.

As a limitation of the study, it should be noted that the proposed features were tested only in highly controlled conditions of isometric contractions. The experiments during non-isometric contractions should be performed in order to validate the quality of the features in dynamic and more natural movements. Also, the experiment included only four tasks related to the elbow joint. Further analysis should include higher number of more complex tasks related to hand and shoulder. Moreover, all results were obtained during offline analysis. To evaluate practical aspects of the features, the experiment should be repeated using online identification and considering multiple transitions between tasks.


%\section{Conclusions}
In conclusion, a new set of features for the identification of isometric motor tasks of upper limb was proposed. It was based on the combination of intensity and the spatial distribution of intensity of HD-EMG. These new features were evaluated using the LDA classifier and the results showed they improve the identification of tasks. Moreover, robustness of the features was tested under the influence of slow time-dependent changes of the EMG. They proved to be particularly useful for task identification when muscles were fatigued. The proposed methods could be used for the design and monitoring of rehabilitation therapies intended for patients with neuromuscular impairment, as well as for the control of external devices like exoskeletons, and prostheses.


\section{Main contributions}

The original contributions provided by the compendium of publications of this thesis are:

\begin{itemize}
\item The definition of a novel algorithm for task and force identification, and its validation in terms of robustness during slow time dependent changes, such as fatigue and drying of conductive gel, on patients with incomplete spinal cord injury. Specific co-activation pattern exists both in intensity and spatial distribution for each patient.  

\item The co-activation pattern in intensity and its spatial distribution of HD-EMG was identified for the group of patients with spinal cord injury. Furthermore, greater similarity was found within the group of patients with similar level of injury. This result implies the possibility of building assistive/rehabilitation device for the group of patients with significantly lower training time. 

\item Novel spatial feature derived from the HD-EMG was used for identification of task and effort level. This feature is based on the probability density function of the HD-EMG activation map. 


\end{itemize}

\section{Future Work}
\label{sec:fw}
The work developed in this thesis open new possibilities in the brain research line of the \emph{BIOsignal Analysis for Rehabilitation and Therapy Research Group (BIOART)} to which the candidate belongs. Some of the most interesting further possibilities are the following: 

\begin{itemize}
\item 

\item 

\item 
\item  

\item 

\item 

\item 
\end{itemize}

\section {Publications derived from the thesis}
\subsection{Journal papers}

\begin{itemize}
\item Jordanić, M., Rojas-Martínez, M., Mañanas, M.A., Alonso, J.F., Marateb, H.R. A Novel Spatial Feature for the Identification of Motor Tasks Using High-Density Electromyography. \textit{Sensors}, 17(7): 1597, 2017, JCR 2.077, Q1 in Instruments and instrumentation (10/58)

\item Rojas-Martínez, M., Alonso, J.F., Jordanić, M., Romero, S., Mañanas, M.A. Identificación de tareas isométricas y dinámicas del miembro superior basada en EMG de alta densidad. \textit{Revista Iberoamericana de Automática e Informática Industrial}, Accepted for publication 2017, JCR 0.390, Q4 in Automation and Control Systems (57/60)

\item Jordanić, M., Rojas-Martínez, M., Mañanas, M.A., Alonso, J.F. Prediction of isometric motor tasks and effort levels based on high-density EMG in patients with incomplete spinal cord injury. \textit{Journal of Neural Engineering}, 13(4): 46002, 2016, JCR 3.465, Q1 in Biomedical Engineering (13/77)

\item Jordanić, M., Rojas-Martínez, M., Mañanas, M.A., Alonso, J.F. Spatial distribution of HD-EMG improves identification of task and force in patients with incomplete spinal cord injury. \textit{Journal of NeuroEngineering and Rehabilitation}, 13(1): 41, 2016, JCR 3.222, Q1 in Rehabilitation (3/65)
\end{itemize}




