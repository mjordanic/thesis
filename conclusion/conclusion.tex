\chapter{Conclusion}
\label{ch:conclusions}

\section{Summary}

Task identification and movement estimation based on EMG are very popular topics involving different areas in machine learning, and, particularly pattern recognition with many possible applications in assistive and rehabilitation devices. The emergence of high-density EMG (HD-EMG) opened new possibilities for extracting neural information and it has been reported that spatial distribution of HD-EMG intensity is a valuable feature in identification of isometric tasks.

This doctoral thesis investigates further the spatial muscle co-activation patterns of myoelectric activity extracted from the HD-EMG activation maps. HD-EMG was measured on five muscles of forearm and upper arm in monopolar configuration during isometric forearm tasks. Measurements were performed on the group of healthy subjects and on the group of patients with incomplete spinal cord injury.

In the chapters 3 and 4, co-activation patterns of patients with incomplete spinal cord injury were analyzed by the means of pattern-recognition-based identification of task and effort level. Both intensity related features, and spatial features were analyzed. In chapter 3, co-activation patterns were analyzed for each patient individually, whereas in chapter 4, co-activation patterns were analyzed within the group of patients. In spite the great diversity between different patients and their levels and types of injury, similarities between activation patterns were found not only in intensity of myoelectric signal, but also in spatial distribution expressed as center of gravity.

In the chapter 5, novel feature for task identification was proposed. The feature is based on spatial distribution of myoelectric activity recorded by HD-EMG. This new feature was evaluated in identification of task and identification of task and effort level in healthy subjects. The evaluation was performed for each subject individually. 


\section{Main conclusions}

In chapter 3, muscle co-activation patterns were analyzed during four isometric tasks and three effort levels in patients with incomplete spinal cord injury. Intensity-based activation maps were calculated for each muscle and different features were extracted: the average intensity of an HD-EMG map, and the center of gravity of an HD-EMG maps. Using the extracted feature sets, a successful patient-specific task identification method was designed. It is capable to estimate with high accuracy not only the motor task, but also the force. This implies that patients with incomplete spinal cord injury have repeatable co-activation muscular pattern not only in intensity, but also in spatial distribution of intensity over the muscle surface. Moreover, the results lead to the conclusion that spatial distribution of myoelectric activity has significant and discriminative power in classification. Furthermore, adding information on spatial distribution of myoelectric intensity improves not only identification result, but also resilience to fatigue and time effect.

Furthermore, in chapter 4 it was discovered that the repeatable patterns in intensity and spatial distribution besides existing for each patient individually, also exist for the entire group of patient. To demonstrate the existence of distinguishable group-specific patterns in HD-EMG, the identification of different tasks was performed, where classifier was not trained exclusively using the samples of a single patient, but it was trained using the samples of all patients, and tested using the samples of all patients, i.e., group-specific classifier was designed. The existence of the patterns is an interesting result because there should be a high level of variability between patients due to the nature of the injury. Co-activation patterns were found not only between different tasks, but also between different effort levels. Group-specific identification of motion intention in patients with neuromuscular impairment could potentially improve the translation of pattern recognition techniques to clinical practice. Also, the results show that the similarity is greater between patients with similar level of lesion (vertebra at which spinal cord injury occurred). This could also have an interesting implication in translation to the clinical practice because patients with the similar level of injury could be able to use the same assisitive/rehabilitation devices with greater ease. 

Finally, in chapter 5, a novel feature for identification of task and effort level was designed. It is based on the locations of local maxima of the probability density function of HD-EMG activation maps, and is obtained using the mean shift algorithm. The feature was tested on the population of healthy subjects in subject-specific approach, that is, classifier was trained and tested for each subject individually. The feature yields higher identification indices compared to the more classical features, especially in task identification at very low effort level. By analyzing the influence of fatigue and other time-dependent changes (e.g. drying of conductive gel) on identification, novel feature had a very good performance. Since the goal of this study was to analyze different feature sets rather than classification methods, LDA was utilized given that this method is the most commonly used, and is generally recommended for myoelectric interfaces \citep{Hakonen2015}.

Density function from which modes were extracted represents RMS activation maps of the HD-EMG. Although the feature proved to be useful, by calculating RMS value, the information is partially lost. Therefore, the modes, or other statistical measures of the raw HD-EMG, i.e. joint distribution of instantaneous EMG amplitude over the electrode array, could also be a useful feature in identification of motion intention. Furthermore, in the literature, features are often calculated for each channel separately and then selected prior the classification using the, e.g., sequential method \citep{Hargrove2009, Li2017}, selection based on common spatial patterns \citep{Geng2014}, or based on the independent component analysis clustering \citep{Naik2016}. Modes of the HD-EMG density function could be correlated with the channels with discriminative information and could be a useful tool in channel selection.

Finally, the mean shift algorithm can be used for clustering and, since it was shown that the algorithm is most effective in low-dimensional data, image segmentation is one of its most successful applications \citep{Comaniciu2002}. A mode of the density estimate, or in this case, a channel selected by the mean shift algorithm, can be considered as a cluster representative \citep{Hennig2015}, related to the possible image segments, where spatial (pixel locations) and range features (the intensity of the grayscale value) are considered. The advantage of the mean shift is that it can be used for clustering non-convex shapes, albeit, it could segment complex non-convex regions in the activation maps. Since segmentation of the muscle activation map can improve the neuromuscular activity estimation \citep{Vieira2010}, this could be a reason why mean shift features improved the performance of the movement detection system compared with previously published attributes. In addition, the algorithm only requires setting one parameter, bandwidth ($h$) and, unlike in the similar methods, it is not necessary to define the number of expected clusters. This is a big advantage because it does not require a priori knowledge on the number of clusters.

The proposed motor task identification method based on spatial information of myoelectric distribution could contribute to the human-machine interface technology. There are many possible applications for this type of technology, for example computer games, exoskeletons, automatic wheelchairs, rehabilitation robots, prostheses, etc. Nowadays, field of brain-computer interface (BCI) technology is advancing very fast with high investments of leading global corporations. However, non-invasive BCI is still an open problem with low output rate, which can be greatly improved by using EMG-based identification of motor intention. For example, 
Müller-Putz et al. suggest non-invasive hybrid brain-computer interfaces (hybrid BCI) designed as EEG-based system, supplemented with other biological and mechanical signals  \citep{Muller-Putz2015}. Joining EEG and EMG recordings in identification of task intention significantly improves the accuracy of individual EEG or EMG system. EMG usually has higher SNR ratio than EEG and it is widely used in the identification of the motion intention, however, it is prone to malfunction due to fatigue. When fatigue occurs, the supplemented EEG input keeps the identification stable, and increases the robustness of the system. Thus, advances in obtaining methods more robust to fatigue or time effect are very interesting.

Some patients with neuromuscular impairment can weakly activate their muscles, but insufficiently to generate a movement. In these patients, as well as in patients that can generate only weak movements, HD-EMG maps can still be generated and used in identification of motion intention, as demonstrated in this study. This approach could supplement the existing BCI or inertial sensors based prostheses and result in a device with a better performance. For example, \citet{Rohm2013} performed a very interesting study with a single SCI patient. Their neuroprosthesis consisted of a functional electrical stimulation of the forearm and upper arm muscles, and a semiactive elbow orthosis. Using BCI and a shoulder joystick, the patient was able to perform complex hand and elbow tasks from everyday life (e.g. eating an ice cream cone). The reported performance of that study was 70\%, which was remarkable considering the fact that the patient did not have any control over involved muscles. However, performance of similar patients could be increased using hybrid BCI if myoelectric activation exists.

As a limitation of the thesis, it should be noted that the proposed features were tested only in highly controlled conditions of isometric contractions. The experiments during non-isometric contractions should be performed in order to validate the quality of the features in dynamic and more natural movements. Also, the experiment included only four tasks related to the elbow joint. Further analysis should include higher number of more complex tasks related to hand and shoulder. Moreover, all results were obtained during offline analysis. To evaluate practical aspects of the features, the experiment should be repeated using online identification and considering multiple transitions between tasks.


\section{Main contributions}

The original contributions provided by the compendium of publications of this thesis are:

\begin{itemize}
\item The definition of a novel pattern-recognition algorithm for task and force identification. The method was based on combination of intensity and spatial distribution of intensity of myoelectric signal, namely center of gravity. The algorithm was validated in the group of patients with incomplete spinal cord injury in terms of robustness during slow time dependent changes, such as fatigue and drying of conductive gel. The results prove the existence of repeatable co-activation pattern in intensity and spatial distribution for each patient. Furthermore, the pattern exists for different tasks, but also for different effort levels.

\item The co-activation pattern in intensity and its spatial distribution of HD-EMG was identified for the group of patients with spinal cord injury. After the injury there is a coherence between activation patterns of different patients, both task-related and force-related. This coherence can be observed in intensity of HD-EMG, but also in spatial distribution of intensity. Furthermore, greater similarity was found within the group of patients with similar level of injury. This result implies the possibility of building assistive/rehabilitation devices for the group of patients with significantly lower training time.

\item Definition of novel statistical spatial feature derived from the HD-EMG. It was used for the identification of task and effort level in group of healthy subjects. This feature is based on the probability density function of the HD-EMG activation map. 


\end{itemize}

\section{Future Work}
\label{sec:fw}
The work developed in this thesis opens new possibilities in the field of myocontrol in rehabilitation. Some of the most interesting possibilities for future works are the following: 

\begin{description}
\item[Dynamic contractions] \hfill \\ 
	The use of spatial information of myoelectric activity is a novel method which already showed very good results in identification of tasks, both in healthy subjects, and in patients with incomplete spinal cord injury during isometric contractions. Isometric contractions are standard to the field of work, that is, pattern recognition for control of human-machine interfaces, and are a good starting point to test the new feature with respect to more classical features. Recordings during isometric contractions provide measurements with more controlled conditions, i.e., minimized influences related to relative shift of recording electrodes with respect to source of the signal – muscle fiber. Therefore it is a good practice to start using new features in graduate analysis in order to establish reliably and precisely the circumstances in which features are useful. However, further studies are necessary to consider non-isometric contractions, which are closer to real conditions. One of this study was already performed within the scope of the thesis and the results were published:\\
	\small{Rojas-Martínez, M., Alonso, J.F., Jordanić, M., Romero, S., Mañanas, M.A. \textbf{Identificación de tareas isométricas y dinámicas del miembro superior basada en EMG de alta densidad}. \textit{Revista Iberoamericana de Automática e Informática Industrial}, Accepted for publication 2017, JCR 0.390, Q4 in Automation and Control Systems (57/60)}
	
	
%	However, isometric contractions The  the merit of combination of intensity and spatial features for identification of specific group of tasks during isometric contractions, and extends the analysis to identification of dynamic tasks. Research is focused on control strategies of upper-limb in normal subjects. Tasks analyzed in the study are related to hand movement in horizontal plane, which mostly involves movements at shoulder joint. These movements were found important because they are commonly used in rehabilitation robots. Although this research was performed on healthy subjects, methods which are proposed could be used for the design and monitoring of rehabilitation therapies intended for patients with neuromuscular impairment. 
	
%	We are already planning and developing the new framework within which we will record non-isometric upper-limb tasks. In our opinion, real time identification of non-isometric tasks should be the final goal of the project.

\item[Generalized mean shift approach] \hfill \\
	In chapter 5 is explained the motor task identification algorithm that uses the novel spatial feature. This spatial feature is based on the modes of the probability density function of HD-EMG activation maps. Instead, the viability of features based on the  modes of the probability density function of raw HD-EMG signal should be explored. Since the information is partially lost by calculating the RMS value of the signal to obtain the activation maps, using joint distribution of instantaneous EMG amplitude over the electrode array could provide higher identification results.
	
\item[Mean shift approach for channel selection] \hfill \\
	Geng et al. recently proposed a more advanced channel selection method based on common spatial patterns \citep{Geng2014} and Naik et al. propose the channel selection based on the independent component analysis \citep{Naik2016}. Modes of the HD-EMG density function, a novel feature proposed in chapter 5, could be correlated with the channels with discriminative information and could be a useful tool in channel selection.

\item[Real time application] \hfill \\
	The task identification system cannot find application without ability of online processing. Therefore, appropriate recording device along with an optimized processing unit should be built. The device should be able to process the task identification in real time using optimized firmware.

\item[Hybrid brain-computer interface] \hfill \\
	The fusion of EEG and EMG could further improve the results of upper-limb task identification, the study we performed using only HD-EMG recordings both in healthy subjects and iSCI patients. This type of study can have impact on numerous fields of application including brain – computer interfaces (BCI). A goal could be to exploit the fusion of cerebral and neuromuscular information and to quantify the improvements when the innovative technique of HD-EMG is joined with the cerebral activity, what was recently called by the research community a \emph{hybrid BCI} \citep{Muller-Putz2015, Rohm2013}.

\item [Increase of identification fidelity] \hfill \\
	Fidelity of the identification could be increased further by using an adaptive model of classifier that is being constantly updated throughout the exercise in order to compensate for the changes in the myoelectric signal caused by, e.g., fatigue ore sweating. There are several recent publications on this subject \citep{Hahne2015, Vidovic2016, Sensinger2009}.

\item[Spatial distribution of frequency] \hfill \\
	Features extracted from frequency/scale domain proved to be very useful in identification of motor task \citep{Oskoei2007}. In future works, it would be interesting to investigate the spatial distribution of frequency over the muscle in search of the discriminative feature.

\end{description}

\section {Publications derived from the thesis}
\subsection{Journal papers}

\begin{itemize}
\item Jordanić, M., Rojas-Martínez, M., Mañanas, M.A., Alonso, J.F., Marateb, H.R. A Novel Spatial Feature for the Identification of Motor Tasks Using High-Density Electromyography. \textit{Sensors}, 17(7): 1597, 2017, JCR 2.077, Q1 in Instruments and instrumentation (10/58)

\item Rojas-Martínez, M., Alonso, J.F., Jordanić, M., Romero, S., Mañanas, M.A. Identificación de tareas isométricas y dinámicas del miembro superior basada en EMG de alta densidad. \textit{Revista Iberoamericana de Automática e Informática Industrial}, Accepted for publication 2017, JCR 0.390, Q4 in Automation and Control Systems (57/60)

\item Jordanić, M., Rojas-Martínez, M., Mañanas, M.A., Alonso, J.F. Prediction of isometric motor tasks and effort levels based on high-density EMG in patients with incomplete spinal cord injury. \textit{Journal of Neural Engineering}, 13(4): 46002, 2016, JCR 3.465, Q1 in Biomedical Engineering (13/77)

\item Jordanić, M., Rojas-Martínez, M., Mañanas, M.A., Alonso, J.F. Spatial distribution of HD-EMG improves identification of task and force in patients with incomplete spinal cord injury. \textit{Journal of NeuroEngineering and Rehabilitation}, 13(1): 41, 2016, JCR 3.222, Q1 in Rehabilitation (3/65)
\end{itemize}


\subsection{Conference papers}

\begin{itemize}
\item Jordanić, M., Rojas-Martínez, M., Mañanas, M.A. Muscle pattern from HD-EMG applied to identification of movement intention. Summer School on Neurorehabilitation (SSNR 2015), 2015, Valencia, Spain

\item   Jordanić, M., Rojas-Martínez, M., Mañanas, M.A., Alonso, J.F. Use of frequency features of HD-EMG in identification of upper-limb motor task. \textit{Cognitive Area Networks}, 4(1): 19:23, 9. Simposio CEA de Bioingeniería: Interfaces Cerebro-Máquina y Neurotecnologías para la Asistencia y la Rehabilitación, 2017, Badalona, Spain

\item Jordanić, M., Rojas-Martínez, M., Alonso, J.F., Migliorelli, C. Mañanas, M.A. Identificación de Contracciones Isométricas de la Extremidad Superior en Pacientes con Lesión Medular Incompleta mediante Características Espectrales de la Electromiografía de Alta Densidad (HD-EMG). Jornadas de Automática (Bioineniería), 2017, Gijon, Spain

\end{itemize}


