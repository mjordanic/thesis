\chapter{Conclusion}
\label{ch:conclusions}

\section{Summary}

Magnetoencephalography is a noninvasive technique that provides excellent temporal resolution and a whole-head coverage that allows the spatial mapping of cerebral sources. These characteristics make MEG an appropriate technique to localize the epileptogenic zone (EZ) in the preoperative evaluation of pharmacoresistant epilepsy. There is great interest in using interictal biomarkers to determine the EZ because these can occur more frequently than seizures \citep{Staba2014} and reduce the discomfort of the patient during the recordings. 

Presurgical evaluation with MEG can guide the placement of invasive electrodes, the current gold standard in the clinical practice, and even supply sufficient information for a surgical intervention without invasive recordings, reducing invasiveness, discomfort, and cost of the presurgical epilepsy diagnosis \citep{Aydin2015}. However, MEG signals have low signal-to-noise ratio compared with intracranial EEG and can sometimes be contaminated by noise that mask or distort the brain activity. This may prevent the detection and the localization with interictal epileptiform discharges (IEDs) and high frequency oscillations (HFOs), two important biomarkers used in the preoperative evaluation of epilepsy. 

To achieve the main goal of this thesis, namely the development and validation of methods for noninvasive localization of interictal biomarkers with MEG, its signal-to-noise ratio must improve. The reduction of two kinds of interference was aimed: metallic artifacts that affect exclusively MEG recordings and mask the activity of IEDs; and the high-frequency noise, produced mainly by muscular interferences that mask HFO activity. Considering the large number of MEG channels and the long time of the recordings, reducing noise and marking events manually is a time-consuming task. The algorithms presented in this thesis provide automatic solutions aimed at the reduction of interferences and, the detection of HFOs. 

In chapter \ref{ch:3}, an automatic BSS-based algorithm to reduce metallic interference was developed and validated using simulated and real signals \citep{Migliorelli2015}. Three methods were evaluated: AMUSE, a second-order BSS technique; and INFOMAX and FastICA, based on high-order statistics. To objectively evaluate the effectiveness of BSS and the subsequent interference reduction, simulated signals consisting in real artifact-free data mixed with real metallic artifacts were generated. Subsequently, an automatic detection of the artifactual components was proposed, exploiting the known characteristics of metallic-related interferences. Results indicated that AMUSE performed better when recovering brain activity and allowed an effective removal of artifactual components.

In chapter \ref{ch:4} the influence of metallic artifact filtering using the previously developed algorithm was evaluated in the source localization of IEDs in patients with refractory focal epilepsy \citep{Migliorelli2016}. In this study, a comparison between the resulting positions of equivalent current dipoles (ECDs) produced by IEDs was performed: without removing metallic interference, rejecting only channels with large metallic artifacts and after BSS-based reduction. The results showed that a significant reduction on dispersion was achieved using the BSS-based reduction procedure, yielding feasible locations of ECDs in contrast to the other two approaches. 

In chapter \ref{ch:5} an algorithm for the automatic detection of epileptic ripples in MEG using beamformer-based virtual sensors was developed \citep{Migliorelli2017}. The automatic detection of ripples was performed using a two-stage approach. In the first step, beamforming was applied to the whole head to determine a region of interest where more high frequency activity was taking place. In the second step, the automatic detection of ripples was performed using the time-frequency characteristics of these oscillations. The performance of the algorithm was evaluated using intracranial EEG recordings as gold standard. Furthermore, the number of events that were detected inside the region of interest were significantly higher than the number of events found outside. 

The main conclusions of the different studies and its relationship with the objective of the thesis are summarized in the following section.

\section{Main conclusions}

%\item	To develop algorithms using freely available signal analysis toolboxes to improve signal-to-noise ratio of MEG recordings in low and high frequencies. 
The main objective of this thesis was to develop and validate methods for the effective noninvasive localization of interictal epileptogenic biomarkers with magnetoencephalography. Two automatic algorithms were developed: a BSS-based procedure to reduce metallic interference present in MEG signals coming from diverse sources; and a method to reduce high frequency interferences and to detect ripples in non-invasive data. Both algorithms were developed using freely available signal analysis tools and methods. Specifically, the toolboxes that were used to achieve this aim were the Fieldtrip toolbox \citep{Oostenveld2011}, Brainstorm software \citep{Tadel2011}, ICALAB \citep{Cichocki2003} and the openly-available S-transform algorithm \citep{Stockwell1996}. 

%\item	To automatically reduce metallic artifacts in MEG recordings using blind source separation techniques evaluating the improvement of signal-to-noise ratio quantitatively. --> hablar de señales simuladas en un parrafo y otro parrafo del TSSS. 
%\item   To compare the performance of different BSS algorithms in the reduction of metallic interference. --> Hablar de amuse, fast-ica infomax

Metallic artifacts can hugely distort MEG signals and mask the brain activity partially or even completely. The proposed algorithm in chapter \ref{ch:3} extracted independent components (ICs) and automatically selected those ones that were artifact-related \citep{Migliorelli2015}. To do so, two characteristics were taken into account both related to metallic artifact nature: the first one measured the frequency spectrum of the IC and the second the regularity of the signal using the sample entropy. Furthermore, an additional condition related to the region where this component projected was taken into account. To validate the automatic algorithm for metallic reduction presented in, simulated signals were generated, consisting of clean recordings to which metallic artifacts were added. The use of simulated signals allowed to objectively quantify the amount of noise reduction achieved by the algorithm using three different approaches: AMUSE, INFOMAX and FastICA. The algorithm presented good performance with the AMUSE technique and was able to remove the metallic interference coming from different sources, projecting in diverse locations and with various levels of intensity. 

AMUSE is a second-order statistics BSS method while FastICA and INFOMAX are high-order statistics techniques. AMUSE showed considerably lower errors than the HOS techniques. INFOMAX was able to decompose artifactual ICs from those related to brain activity but part of the cerebral signals remained mixed with the metallic components. FastICA did not manage to generate a valid decomposition of the ICs, even after ensuring convergence with the stabilizer provided by \citep{Hyvarinen1999}. The reason behind this behaviour is that FastICA decomposes signals by measuring their differences with a normal distribution. For this reason it is very effective dealing with supergaussian sources, such as cardiac and ocular interferences, but when working with subgaussian and specially with gaussian signals which is the case of metallic interferences, the algorithm is not able to separate them from cerebral signals, that present similar distributions. 

Other methods that deal with metallic interference are the so-called signal space separation (SSS) \citep{Taulu2004} and temporal signal space separation (tSSS) \citep{Taulu2006}. However, these algorithms are provided by a registered software only available for Elekta-Neuromag systems \citep{GonzalezMoreno2014}. The developed BSS-based approach is based on standard libraries that can be used in any MEG system available. Moreover, while tSSS algorithm was evaluated in the removal of metallic interference in previous studies, \citep{Hillebrand2013,Song2009,Alexpoulos2013} these studies did not evaluate the noise reduction of metallic interference quantitatively. Furthermore, in \citep{GonzalezMoreno2014} a higher signal-to-noise ratio improvement was observed when applying BSS techniques combined with SSS and tSSS methods in comparison of applying these techniques alone, suggesting that after applying SSS and tSSS techniques not all the noise was successfully removed and BSS algorithms were effective in removing the remaining interferences. 

%\item   To evaluate the influence of metallic interference and its reduction in the detection and localization of interictal epileptiform discharges. --> hablar de los resultados del 2º paper
The influence of metallic interference and the subsequent reduction with the developed algorithm was tested in patients with pharmacoresistant epilepsy in \citep{Migliorelli2016}. All of the patients had some kind of non-removable device that produced metallic interferences: dental orthodontics, implanted subdural grids, vagus nerve stimulators, or ventricular bypass valves. For these patients, IEDs were detected in the raw recordings by three different experts and the estimation of the ECDs was performed in three different scenarios: without removing the interference, discarding the most affected channels, and applying the automatic removal procedure. For the three scenarios several measures were used to evaluate the ECD estimation: the confidence volume, the distribution of the dipoles at the onset site, and the running distance of the path followed by dipoles associated to each IED.

In all the subjects, using the BSS-based approach produced smaller confidence volume values for the IED detection, suggesting that each of the computed dipoles and the obtained models were more plausible. Furthermore, the dispersion of the locations of the selected IEDs was measured. Using the automatic BSS technique, the dipoles showed significantly lower dispersion values than with the other proposed approaches, providing a better delineation of the irritative zone, which is the area generating IEDs. Finally, a distance measure allowed to observe the improvement on the stability of the dipolar sources and therefore a more reasonable association with the discharges produced by the epileptic tissue. 

%\item   To reduce high-frequency noise to detect high frequency oscillations in MEG signals. --> hablar de beamformer y los resultados visuales del 3º paper
%\item	To develop an automatic method to detect high frequency oscillations in MEG signals using intracranial recordings as \emph{gold standard}. -->
In addition to the detection of IEDs, and in line with the main objective, the thesis tackled the detection of high-frequency oscillations in MEG signals. These oscillations often appear masked by the high-frequency activity. In chapter \ref{ch:5}, an algorithm was proposed based on spatial filtering beamforming to reduce very noticeably the high-frequency noise, allowing the visual observation of ripples that could not be identified in scalp MEG. Furthermore, the algorithm allowed to automatically detect ripples and discard other noisy events just by using a two-step procedure based on thresholds. The validation of the algorithm was performed using the time-domain information provided by intracranial electroencephalographic recordings (iEEG), the current gold standard. It is important to remark that the automatic detection of HFOs in iEEG is still an open field of study, and that MEG and iEEG are not supposed to record exactly the same type of cerebral activity. While iEEG provides an excellent spatial resolution in which electrodes are implanted, MEG can provide a whole-head image of the network areas that are activated during high-frequency epileptogenic activity.  


%\item	To automatically localize the areas generating pathologic high frequency oscillations.
The results showed that the proposed automatic selection of the area generating most events of interest agreed with the affected zone targeted by clinicians. Moreover, the virtual sensors showing higher ripple activity inside this area appeared focalized in comparison with its contralateral lobe. This finding is in correspondence with the observations of other studies that suggest that the areas producing pathological ripples are less extended than the areas generating physiological HFOs \citep{Chrobak1996,Bragin2002}. Furthermore, although the automatic detection algorithm uses a delimited region to detect ripples, this is not a limitation of the method because once these are detected, they can be mapped \emph{a posterori} throughout the head in order to observe the brain networks activated during high frequency epileptogenic activity. 

%\item	To publish the obtained results and conclusions in high-impact journals, as well as in international and national conferences.
Finally, these findings were published in high-impact journals and they are part of this compendium of publications. In addition, the preliminary studies on the removal of metallic interference and on the detection of high-frequency oscillations were published in the $35^{th}$ Annual International Conference of the IEEE Engineering in Medicine and Biology Society \citep{Migliorelli2013} and in the International Conference of Neurorehabilitation \citep{Migliorelli2017b} respectively. Therefore, it can be concluded that the specific objectives raised in this thesis: (I) to develop algorithms, using freely-available signal analysis toolboxes, (II) to compare the performance of different BSS algorithms in the reduction of metallic interference, (III) to reduce metallic artifacts in MEG recordings automatically using blind source separation techniques, (IV) to evaluate the influence of metallic interference in the localization of interictal epileptiform discharges, (V) to reduce high-frequency noise to detect high-frequency oscillations in MEG signals, (VI) to develop an automatic method to detect high-frequency oscillations in MEG signals, and to compare its detection with intracranial recordings, (VII) to localize the areas generating pathologic high frequency oscillations automatically, and (VIII) to publish the obtained results and conclusions in high-impact journals and conferences;  were accomplished allowing the fulfillment of the main objective of this thesis: to develop and validate methods for the effective noninvasive localization of interictal epileptogenic biomarkers with magnetoencephalography. 


\section{Main contributions}

The original contributions provided by the compendium of publications of this thesis are:

\begin{itemize}
\item The definition of a novel algorithm, and its validation with simulated signals to remove metallic interference from MEG recordings in a fully automated fashion. This algorithm was developed using freely available toolboxes and methods and can be used on any MEG system. 

\item The evaluation of the effects of the automatic filtering procedure in the detection of IEDs and the estimation of ECDs of patients with focal refractory epilepsy. The reduction of metallic interference in this kind of patients is essential, because the purpose of MEG is to provide a reliable detection of the areas producing epileptic activity, but there is a high number of patients with irremovable metallic devices (metallic intracerebral electrodes, vagal stimulators or, in younger patients, dental orthodontics) that distort the signals and produce noneffective localization of the IEDs generators.

\item The definition of a novel algorithm, and its validation with intracranial recordings, to reduce high-frequency noise and to automatically detect HFOs, a promising interictal biomarker of epilepsy strongly linked to the epileptogenic zone. To date, this is the only fully-automatic algorithm that detect these oscillations at the source level in noninvasive recordings. 

\item The previous contributions allow an improved noninvasive detection and localization of interictal epileptic biomarkers, which can help in the delimitation of the epileptogenic zone and guide the placement of intracranial electrodes, or even to determine these areas without additional invasive recordings. As a consequence of this improved detection, and given that interictal biomarkers are much more frequent and easy to record than ictal episodes, the presurgical evaluation process can be more comfortable for the patient. 

\end{itemize}

\section{Future Work}
\label{sec:fw}
The work developed in this thesis open new possibilities in the brain research line of the \emph{BIOsignal Analysis for Rehabilitation and Therapy Research Group (BIOART)} to which the candidate belongs. Some of the most interesting further possibilities are the following: 

\begin{itemize}
\item The growing use of brain stimulation in research and clinical applications reflects its capabilities to modulate brain function in ways not feasible with other techniques \citep{Peterchev2012}. However, large artifacts can appear in MEG and EEG recordings when used simultaneously with this technique \citep{Oswal2016,Haumann2016}. The BSS-based automatic algorithm for the reduction of metallic interference can be useful to deal with this source of artifacts and provide new solutions to observe the brain activity that can stay masked behind large sources of interference.

\item The improvement of the beamformer-based algorithm for the detection of HFOs in MEG by using time-frequency characteristics and advanced clustering methods \citep{Liu2016} in the whole-head virtual sensors would allow to  identify the epileptogenic zone in an automatic, accurate, and efficient manner. This, in turn, would allow to characterize the spatial, frequency, and temporal distribution of the high-frequency activity. 

\item Increased efforts should be put to improve the SNR of the noninvasive cerebral signals, and in this sense independent component analysis has proven to be effective in improving the spatial localization of beamformer spatial filters \citep{Fatima2013}. These techniques could be combined in order to improve the localization of HFOs and IEDs. 

\item  Due to the sampling rate limitation of the available MEG database, the beamformer-based automatic detection algorithm only evaluated ripples up to 120Hz. However, the algorithm does not depend on the sampling rate and can be used to detect higher frequency HFOs content if signals are acquired at higher rates. Several studies have found that fast ripples have shown to be more correlated with the seizure onset zone, and presumably more epileptogenic \citep{Jacobs2008,vanKlink2014,Zijlmans2012} than ripples. The application of the automatic algorithm to databases with higher frequency rates would allow the comparison between ripples and fast ripples in terms of density or spatial location. 

\item Although the results show a significant improvement of the localization of IEDs and HFOs with noninvasive techniques, additional validation taking into account the spatial localization of the epileptic focus provided by ictal intracranial recordings would be helpful. However, this information was not available for the recordings used for this thesis. Also, an interesting study combining multimodal analysis techniques such as EEG or fMRI would provide more information about the localization of the cerebral sources.

\item Using the developed algorithms, the spatial comparison of the brain areas generating IEDs and HFOs could be objectively performed. This will allow to identify differences between the IEDs originating in the same areas than HFOs, and those generating in widespread areas. Understanding the differences between IEDs and its relationship with HFOs would help to interpret the processes of epileptogenesis and to determine whether there are different types of IEDs, some that might act to suppress seizures, while others could promote them \citep{Staba2011}.

\item Finally, to faster and increase the use of the methods developed in this thesis, the algorithms could be published, along with other developed by the BIOART group, in an open and freely-available toolbox. This will facilitate the use of these techniques by other research groups or even by clinicians.

\end{itemize}



%\section{Publications derived from this Thesis}
%	\subsection{Journal Papers}
 %   \subsection{International conferences}