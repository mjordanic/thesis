
\chapter{Introduction}





\section{Muscle anatomy}

Movement is achieved using over 600 skeletal muscles, the ones that can be controlled voluntary. 

Smooth muscles - mostly in internal organs

cardiac muscle

Elementary building block of a muscle is muscle cell, or muscle fiber - \emph{myocyte}. Myocytes are ensheated by \emph{endomysium}, a connective tissue that contains nerves and capillaries. They are organized in bundles of 10 to 100, \emph{fascicles}, and surrounded by sheath of connective tissue, \emph{perymisium}. Group of fascicles is finally grouped together and enveloped by \emph{epimysium}, forming a muscle.

Sarcolema is the cell membrane of myoecyte, consisting of a lipid bilayer.

myoplasma
sarcomeres


Smallest functional unit is called motor neuron, and it was firstly defined by Liddell and Sherrington in 1925 \citep{Liddell1925, Sherrington1925}. It is composed of motor neuron with axon and dendrites, and muscle fibers that axon innervates \citep{Duchateau2011}. Since motor neuron with a single action potential usually evokes action potentials simultaneously in all belonging muscle fibers, by observing action potentials of the muscle fibers, information on activity of motor neurons in spinal cord or brain stem can be inferred \citep{Merletti-Farina-book}. % mozda ne treba ovaj citat

Pool of motor neurons that innervates entire muscle generally ranges from ten to thousand, depending on the muscle. \citep{Merletti-Farina-book}.


The central nervous system (CNS) is responsible for processing information received from all parts of the body. The two main organs of the CNS are the brain and the spinal cord and are entirely composed of two kinds of specialized cells: neurons and glia. The brain is the most complex part of the human body and exerts a centralized control over the other organs. Neurons, the basic working units of the brain, are designed to transmit information within the brain to other nerve cells and to communicate with muscles and gland cells. The complex architecture of the brain is built on the extensive number of interconnected neurons sharing information through specialized connections called synapses. This connection allows  neurons to communicate through an electrical or chemical signals, producing ionic currents that generate electric and magnetic fields. 

The CNS is organized in multiple levels, from simple connections between cells to coordinated cell populations, building a complex architecture of interconnected brain regions. The neural processes at this last level are produced by the dynamic coordination of smaller elements. In the cerebral cortex, all this brain activity is summated and its electric and magnetic fields can be measured on the scalp surface. 

\

