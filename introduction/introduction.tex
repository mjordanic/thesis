

\chapter{Introduction}

Performing a movement is a complicated process that involves many physiological entities working in high coherence. It involves bones, tendons, nerves, and many other working in perfect harmony. Even the simplest movements are rarely performed using just one muscle. Everything we do involves high muscular coordination and constant and precise regulation. While standing, muscles of legs and trunk are constantly simultaneously co-contracting, maintaining balance.  
Muscle is a body tissue capable of transforming chemical energy to force. There are several muscle types: smooth, building internal organs, cardiac, building the heart, and skeletal. Only skeletal muscles can be controlled voluntarily and are used in locomotion. They are usually connected to bones with tendons (collagen fibers), as shown in figure \ref{fig:muscle}.

The neurons controlling the movement are organized in hierarchical fashion \citep{Widmaier2014}. In the highest level of hierarchy, the movement is conceived. Here the complex plan of intention is made. Very little is known about the exact location of neurons responsible for this task. Higher centers then transmit this command to the middle level structures, where the task is elaborated. Simultaneously, this middle level neurons receive the information from the receptors in muscles, skin, tendons, and join, but also from the visual system. Planning of the movement that is about to be performed is performed with respect to the space this movement will occupy, and detailed control signals for each muscle involved in the movement are generated. Centers involved in this tasks are located in cerebral cortex, cerebellum, subcortical nuclei, and brainstem. The information is then transmitted to the lowest level of the motor hierarchy: spinal cord and brainstem. Here the information is transmitted over motor neurons to the muscles. The selection of motor neurons involved in the task and timing is performed at this level. Organization and locations of the neural system for motor control can be seen in figure \ref{fig:brain_centers}, whereas overall figure of motor control can be seen in figure \ref{fig:control-merletti}.
\begin{figure*}[t!]
    \centering
    \begin{subfigure}[t]{0.95\textwidth}
        \centering
        \includegraphics[height=3in]{Images/introduction/control.png}
        \caption{}
    \end{subfigure}%
    
    \begin{subfigure}[t]{0.95\textwidth}
        \centering
        \includegraphics[height=3in]{Images/introduction/control2.png}
        \caption{}
    \end{subfigure}
    \caption{Figure describes \textbf{a)} hierarchical organization of neural system for motor control and \textbf{b)} side view and cross section of the brain showing motor control centers. Retrieved from \citep{Widmaier2014}}
\label{fig:brain_centers}
\end{figure*}

\begin{figure}[ht]
\centering
\includegraphics[width=0.45\textwidth]{Images/introduction/control-merletti2.png}
\caption{Figure represents a schematic representation of motor control mechanisms. Idea of a movement is conceived in the brain, and is getting to spinal cord by neural pathways. Motor neurons exiting spinal cord trigger muscle contraction. Simultaneously, sensory information is being transmitted to the higher controlling mechanisms. Retrieved from \citep{Merletti-book}}
\label{fig:control-merletti}
\end{figure}


\section{Muscle physiology}

Elementary building block of a muscle is muscle cell, or muscle fiber - \emph{myocyte}. They are ensheated by \emph{endomysium}, a connective tissue that contains nerves and capillaries. Myocytes are organized in bundles of 10 to 100 fibers, which are called \emph{fascicles}, and they are surrounded by sheath of connective tissue, \emph{perymisium}. Group of fascicles is finally grouped together and enveloped by \emph{epimysium}, forming a muscle. Muscle can be seen in figure \ref{fig:muscle}.
\begin{figure}[ht]
\centering
\includegraphics[width=0.9\textwidth]{Images/introduction/muscle2.png}
\caption{Organization of skeletal muscle with attachment to the bone. Retrieved from \citep{Widmaier2014}}
\label{fig:muscle}
\end{figure}

\emph{Sarcolema} is the cell membrane of myocyte, consisting of a lipid bilayer that contains intracelular liquid, \emph{myoplasma}. In the myoplasma, thin and thick filaments are serially connected, forming \emph{sarcomeres}, which are longitudinally connected in \emph{myofibrils} that extend through entire length of the myocyte. During shortening of muscle fibers, thin and thick filaments of sarcomeres are pulled together by cross-bridges between them. Total shortening of myofibril is summation of shortenings of sarcomeres of which it is composed.

Each motor neuron at the neuromuscular junction innervates several muscle fibers, forming the smallest functional unit called \emph{motor unit}. It was firstly defined by Liddell and Sherrington in 1925 \citep{Liddell1925, Sherrington1925} and is composed of motor neuron with axon and dendrites, and muscle fibers that axon innervates, as seen in the figure \ref{fig:motor units} \citep{Duchateau2011}. Since motor neuron with a single action potential usually evokes action potentials simultaneously in all belonging muscle fibers, by observing action potentials of the muscle fibers, information on activity of motor neurons in spinal cord or brain stem can be inferred \citep{Merletti-Farina-book}. Pool of motor neurons that innervates entire muscle generally ranges from ten to thousand, depending on the muscle \citep{Merletti-Farina-book}.
\begin{figure*}[t!]
    \centering
    \begin{subfigure}[t]{0.49\textwidth}
        \centering
        \includegraphics[height=1.7in]{Images/introduction/one_MU.png}
        \caption{}
    \end{subfigure}%
    ~ 
    \begin{subfigure}[t]{0.49\textwidth}
        \centering
        \includegraphics[height=1.7in]{Images/introduction/two_MU.png}
        \caption{}
    \end{subfigure}
    \caption{Figure shows \textbf{a)} a single motor unit with motoneuron and muscle fibers it innervates, and \textbf{b)} two motor units where it can be seen how muscle fibers of different motor units are intermingled. Retrieved from \citep{Widmaier2014}}
\label{fig:motor units}
\end{figure*}


By the characteristics of muscle fiber, there are three main types of muscle fibers:
\begin{description}
\item[Fast twitch, fatigable fibers (FF, or type IIb):] This fiber type have high levels of ATP (source of energy) for anaerobic energy supply, and are dominantly present in pale muscles. They are of glycolytic type and work well in ischemic or low oxygen conditions. Regarding contraction properties, they are characterized by fast twitch, large forces and high nerve conduction velocity, but they get fatigued faster than the other muscle fiber types. 

\item[Fast twitch, fatigue-resistant (FR, or type IIa):] These are oxidative glycolytic fibers, characterized by fast twitch and are resistant to fatigue. They have intermediate conduction velocity. 

\item[Slow twitch, very resistant to fatigue (S, or type I):] They are slow oxidative fibers and do not work well in low oxygen conditions. They generate small forces, have slow twitch and are characterized by lower nerve conduction velocity. This fiber type is very resilient to fatigue because of high oxidative metabolism and energy efficiency. They are present in high percentage in red muscles, such as soleus.
\end{description}

Muscle fibers innervated by the same motor neuron have similar histochemical and contractile characteristics, and can be said that motor unit is composed of the muscle fibers of the same type.

Force that muscle fibers generate depends on firing frequency of the action potentials (rate coding) innervating the neuromuscular junction, and the recruitment strategy by which the motor units are activated, i.e., the number of activated motor units. Firing frequency and the recruitment strategy depend on the speed and force of contraction. Muscle units with low threshold are activated firstly, resulting in low force and high endurance, i.e., resistance to fatigue. If greater force is required, muscle units with higher threshold that are prone to fatigue are activated \citep{Freund1975, Merletti-book}. This was firstly proposed by Henneman et al. in 1965 \citep{Henneman1965}, who state that order of recruitment of motor neurons is based on size principle, that is, neurons with smaller axons are recruited at lower effort levels and with increase in force, larger motoneurons are recruited. Therefore, S type muscle units, which have the smallest motoneurons are recruited first, followed by FR type units, and finally FF units. The recruitment strategy and resistance to fatigue can be seen in figure \ref{fig:fibers}. 

%\begin{figure}[ht]
%\centering
%\includegraphics[width=0.5\textwidth]{Images/introduction/fiber_distribution.png}
%\caption{Figure illustrates \textbf{a)} diagram of different muscle fibers in muscle cross section, and \textbf{b)} muscle tension produced by recruitment of different types of muscle fiber. It can be noted that type S fibers are activated first and generate low force level, whereas type FF fibers are activated last and generate high forces . Adopted from Widmaier}
%\label{fig:fiber_distribution}
%\end{figure}
%
%\begin{figure}[ht]
%\centering
%\includegraphics[width=0.5\textwidth]{Images/introduction/fiber_fatigue.png}
%\caption{Figure illustrates the time during which specific muscle fibers can remain tension. Adopted from Widmaier}
%\label{fig:fiber_fatigue}
%\end{figure}

\begin{figure*}[t!]
    \centering
    \begin{subfigure}[t]{0.45\textwidth}
        \centering
        \includegraphics[height=4in]{Images/introduction/fiber_distribution.png}
        \caption{}
    \end{subfigure}%
    ~ 
    \begin{subfigure}[t]{0.45\textwidth}
        \centering
        \includegraphics[height=4in]{Images/introduction/fiber_fatigue.png}
        \caption{}
    \end{subfigure}
    \caption{Figure describes characteristics of different types of muscle fibers. In \textbf{a)} is a diagram of different muscle fibers in muscle cross section (top), and muscle tension produced by recruitment of different types of muscle fiber (bottom), whereas in \textbf{b)} is the illustration of the time interval during which specific muscle fibers can remain tension. It can be noted that type S fibers are activated first, generate low force level, and are resistant to fatigue. On the other hand, type FF fibers are activated last, generate high forces, and develop fatigue fastest. Retrieved from \citep{Widmaier2014}}
\label{fig:fibers}
\end{figure*}


\section{Muscle contraction}

Skeletal muscles are activated voluntarily by electro-chemical impulses of motor neurons. The process is described in this chapter in summarized version. For more detailed description, the reader is pointed to medical literature (e.g. {Widmaier2014}).

During the stable state when there are no stimuli, i.e., in the resting state, the interior of the myocyte is at higher electrical potential that the exterior. This difference in potential is usually around 80 mV and it is caused by the higher concentration of positive ions, namely Na+, outside of the sarcolema \citep{Nazmi2016}, as shown in figure \ref{fig:depolarization}.

\begin{figure}[ht]
\centering
\includegraphics[width=0.95\textwidth]{Images/introduction/Depolarization.png}
\caption{Ilustration of depolarization/repolarization of the muscle fiber. Adopted from \citep{Nazmi2016}.}
\label{fig:depolarization}
\end{figure}

Motor neurons transfer nerve impulses that control the muscle from spinal cord to neuromuscular junction. At the nerve endings, action potentials induce the opening of calcium channels, which enables calcium from extracellular fluid to enter axon terminals and trigger the release of the neurotransmitter \emph{acetylcholine}. Acetylcholine is released to the narrow space between the axon and sarcolema of the myocyte, and causes sodium channels in sarcolema to open and allow the flow of Na+ and K+ ions in both directions. Na+ ions now flow into the myoplasma by diffusion due to higher concentration of Na+ ions outside of the membrane, but because of similar gradient, concentrations of the K+ ions don't change a lot. This process causes depolarization of sarcolema during which the outside potential of the muscle cell is at lower voltage than inside potential by around 30 mV. Depolarization is immediately followed by repolarization, a process during which the electrochemical balance and the resting potential of the cell are restored. It is achieved by flushing the Na+ ions outside of the sarcolema by the \emph{ion pump}. The process can be seen in figures \ref{fig:depolarization} and \ref{fig:action_potential_generation}.
  
\begin{figure}[ht]
\centering
\includegraphics[width=0.95\textwidth]{Images/introduction/action_potential_generation.png}
\caption{Illustration of generation of action potential. Retrieved from \citep{Widmaier2014}.}
\label{fig:action_potential_generation}
\end{figure}
  
If the amount of acetylcoline is sufficient for the excitation, depolarization/repolarization wave, that is, action potential, propagates longitudinally from the neuromuscular junction towards the ends of the muscle fiber causing contraction \citep{Henneberg1999}. Speed of action potential propagation is called \emph{conduction velocity} and typically ranges around 4 m/s.

Detailed analysis of muscle physiology can be found elsewhere \citep{Squire1986, Widmaier2014}.



\section{Muscle fatigue}

According to \citet{Widmaier2014}, muscle fatigue is a decline in muscle tension as a result of previous contractile activity. It is also characterized by decreased relaxation rate and lower shortening velocity of muscle fibers. Muscle fatigue is a continuous process that starts at the moment when muscle unit activates. If muscle keeps contracting long enough, eventually it will stop contracting because of electro-physiological inability to maintain the contraction. This moment is called the failure point \citep{DeLuca1984}. The failure point depends on many different factors physiological characteristics, but also on the number of muscle fibers and proportion of Type I/Type II muscle fibers. Muscles with higher proportion of Type I fibers do not fatigue easily and recover sooner than type II fibers. However, type II fibers are able to generate higher forces \citep{Kupa1995}.

Factors causing fatigue can be found in the muscle itself, which is \emph{peripheral fatigue}, but can also originate in central nervous system, in which case it is \emph{central fatigue}.

With respect to the source of impairment, the muscle fatigue can be:

\begin{description}

\item[Peripheral fatigue] \hfill \\
	Peripheral fatigue occurs in muscle itself, when because of electro-chemical imbalance muscle contraction is prevented. There are three main sources of peripheral fatigue:
	\begin{itemize}
		\item During sustained contraction, sarcolema of the muscle fibers become acid, and this accidifation lowers the muscle fiber conduction velocity \citep{DeLuca1984}.
		
		\item High concentration of K+ ions prevents generation of action potentials in muscle fiber \citep{Widmaier2014}. 
		
		\item Buildup of adenosine diphosphate, a byproduct of muscle contraction, slows the rate of cross-bridge cycling, affecting the relaxation, and reducing shortening velocity \citep{Widmaier2014}.
	\end{itemize}

\item[Central fatigue] \hfill \\
	Central fatigue occurs in central nervous system that controls the movement. It is manifested by synchronization of neural spike trains of different motor units. Probably by following the principle that by activating more muscle units simultaneously, total output force of the muscle is higher. 

\end{description}

Muscle fatigue changes characteristics of the myoelectric signal. Due to decrease of muscle fiber conduction velocity, caused by peripheral fatigue, but also due to synchronization of firing times caused by central fatigue, there is a shift of energy in frequency spectrum of myoelectric signal towards lower frequency, as shown in figure \ref{fig:fatigue} \citep{DeLuca1984}. Another indicator of muscle fatigue is increase of amplitude of surface electromyographic signal. This increase occurs due to two main reasons:
\begin{itemize}
\item Tissue between muscle fibers and recording electrodes positioned on the surface of the skin (e.g. fat layers, skin, etc.) have low-pass properties on the propagating electromagnetic wave. Since the power of the propagating wave shifts towards lower frequencies, amplitude of the recorded signal increase.

\item Due to synchronization of firing patterns cause by central fatigue, amplitude of recorded sEMG signal increases. 
\end{itemize} 

\begin{figure}[ht]
\centering
\includegraphics[width=0.45\textwidth]{Images/introduction/fatigue.png}
\caption{Illustration of force and EMG signal recorded during fatiguing exercise (top), and frequency spectra of corresponding EMG signal (bottom) recorded at the beginning of the exercise (a), and at time when force started decreasing (b). Retrieved from \citep{DeLuca1984}.}
\label{fig:fatigue}
\end{figure}

There are many studies exploiting this changes in sEMG signal to estimate and monitor muscle fatigue. Most of the approaches are based on monitoring frequency characteristics of the signal. One of the simplest measures is number of zero-crossings \citep{Hagg1981}, but it is very sensitive to noise. Mean and median frequencies are often used in literature \citep{Lindstrom1977, Merletti1997, Stulen1981}, but also more advanced time-frequency processing methods \citep{Knaflitz1999, Cifrek2000, Georgakis2003, Srhoj-Egekher2011a}.


\section{Surface electromyography}

Muscle unit action potentnial (MUAP) is the combination of action potentials generated by all muscle fibers belonging to that motor unit, whereas myoelectric signal is a superposition of electrical activity (propagating action potentials) produced by all muscle units. 

There are two main types of electromyography: 
\begin{description}
\item[Surface EMG (sEMG)] This is non-invasive type of EMG measurement where electrodes are positioned on the surface of the skin. Two types of electrodes are used: wet electrodes, which are used in combination with conductive gel that provide high signal quality, and dry electrodes, which can be applied directly on the skin. Although wet electrodes are mostly used, signal quality deteriorates during recording because of evaporation of the gel.

\item[Intramuscular EMG (iEMG)] This is invasive type of recording which implies insertion of needle or wire electrodes under the skin \citep{Marateb1999}. This type of recording is used for precise measurement of narrow volume, for example couple of muscle fibers. It has high SNR, but causes discomfort in subjects. It It is often used in clinical practice because it can detect abnormal functionality. For example, action potentials of spontaneously contracting single muscle fibers can be measured. These potentials are an important sign of deinnervation, but cannot be recorded using sEMG \citep{Merletti-book}.
\end{description}

Although iEMG signal usually has higher quality (in terms of signal-to-noise ratio), it was shown that both approaches provide a similar identification rate of upper-arm motor task \citep{Hargrove2007}. Since sEMG is non-invasive it is preferred method, and it will be used in this Thesis.

%Moreover, although often narrow scope of intramuscular EMG can be beneficial, especially in clinical applications regarding activation of single muscle unit, it does not provide information on other parts of the muscle. For that reason, surface EMG can be more appropriate because it simultaneously records action potentials of large muscle area. On the other hand, that can be a serious drawback because if there are several active muscles in small volume, myoelectric activity of both muscles will be recorded. 
Surface electromyographic signal (sEMG) is the sum of the electrical activity of the muscle fibers recorded on the surface of the skin. Since muscle fibers are activated by the impulse train of the innervating motor neurons, i.e. neural drive to the muscle, sEMG is the convolution of motor neuron spike trains by the motor unit action potential recorded on the electrodes \citep{Farina2010, Farina2014}:
\begin{equation}
sEMG(t) = \sum_{i=1}^{M} \sum_{j=-\infty}^{+\infty} MUAP_i(t)\,\, \delta(t-t_{i,j})
\end{equation}
, where $M$ is the number of active motor units, $MUAP_i(t)$ is the action potential waveform of the $i^{th}$ motor unit recorded by the electrodes, and $t_{i,j}$ is the time of the discharge of the $i^{th}$ motor neuron. This model assumes there is no interference and that neuromuscular junction never fails, which is not the case. In the equation, $MUAP_i(t)$ is related to the electrophysiological state of the muscle fiber membranes and conduction properties of the tissue through which the potential propagates, whereas neural information is contained in motor neuron spike trains $\delta(t-t_{i,j})$ \citep{Farina2014b}. With respect to muscle fatigue explained in the previous section, peripheral fatigue affects $MUAP_i(t)$, whereas central fatigue have effect on $\delta(t-t_{i,j})$ term. It is important to notice that following this model, sEMG reflects all motor control information that is present in motor neuron. For that reason, it is more appropriate to extract motor control information carried by motor neurons using sEMG, than directly by invasive measurement of electrical potential of the motor neuron. The advantage of the sEMG is that multiple fibers are activated simultanously, generating bioelectrical signal with relatively high SNR, which can be measured on the surface of the skin. In this context, sEMG can be considered as the amplified neural signal, whereas muscle can be considered as a biological amplifier of nerve activity \citep{Farina2014}. Origin of sEMG signal can be seen in figure \ref{fig:EMG_origin}.
\begin{figure}[ht]
\centering
\includegraphics[width=0.75\textwidth]{Images/introduction/EMG_origin.png}
\caption{sEMG signal is a superposition of motor unit action potentials recorded on the electrodes convoluted by belonging motor neuron spike train. Retrieved from \citep{Farina2014}.}
\label{fig:EMG_origin}
\end{figure}

%Myoelectric signal can be considered as stochastic process with Gaussian probability density function \citep{DeLuca1984, DeLuca1979}.

Depending on number of electrodes used for the recording. the following classification exists: single-channel recording in monopolar mode, single channel recording in bipolar mode, recording using linear electrode array, and high-density EMG, as shown in figure \ref{fig:electrode_types}).
\begin{figure}[ht]
\centering
\includegraphics[width=0.55\textwidth]{Images/introduction/electrode_types.png}
\caption{Four types of recording surface EMG signal: monopolar, bipolar, linear electrode array, HD-EMG. Figure was modified from \citep{Merletti2010}}
\label{fig:electrode_types}
\end{figure}

In single-channel monopolar recording, a single electrode is positioned over the muscle, whereas the reference electrode is positioned over the place that does not generate electrical activity. On the other hand, single-channel bipolar electrode configuration is most often used, in which signal is a difference of potential between two electrodes. These configuration is traditionally preferred because of the lower interference and higher signal-to-noise ration \citep{Merletti-book}. General recommendation is that the inter-electrode distance is around 20 cm \citep{Hermens1999}, but the optimal distance depends on many factors, as briefly explained in \citep{Hakonen2015}. For both monopolar and bipolar single channel recordings it is recommended that the electrodes are positioned between innervation zone and tendon. Exact recommendations can be found in \citet{Hermens1999}.

Linear electrode array consists of multiple electrodes positioned at equal distance along the muscle line, folowing the direction of propagation of action potentials. Measurements recorded using this type of electrodes provide more information on the muscle. For example, it can be used for estimation of conduction velocity, as shown is figure \ref{fig:conduction_velocity} \citep{Merletti-book}.
\begin{figure}[ht]
\centering
\includegraphics[width=0.60\textwidth]{Images/introduction/conduction_velocity.png}
\caption{Estimating conduction velocity using averaged MUAPs recorded using linear electrode array. Figure was retrieved from \citep{Merletti-book}}
\label{fig:conduction_velocity}
\end{figure}

Technological advancement of EMG acquisition systems enables use of high-density electromyography (HD-EMG) \citep{Zwarts2004}. Using an array of closely spaced electrodes organized in a quadrature grid, multiple EMG channels are recorded over the wide area of the muscle. Electrodes used for HD-EMG recording can be seen in figure \ref{fig:electrode}. This type of recording is more reliable because it can record activations in different parts of the muscle and increase redundancy. HD-EMG is the only EMG recording approach that allows insights into spatial distribution of motor units in a muscle. By observing the amplitude or intensity of signals recorded in different channels, it is possible to analyze how different muscle regions activate depending on joint position \citep{Vieira2010}, contraction level \citep{Holtermann2005}, and duration of movement and fatigue \citep{Tucker2009, Staudenmann2014}. Moreover, Zwartz et al. \citep{Zwarts2003} pointed out that single channel EMG disregards important spatial aspects of MUAP propagation, which are essential for the force-generating capacity of the muscle, and, if not well addressed, can lead to incorrect conclusions. Moreover, since muscles do not activate homogeneously, single bipolar channel EMG has some serious drawbacks, which can be overcome by using 2D electrode arrays. 
\begin{figure}[ht]
\centering
\includegraphics[width=0.5\textwidth]{Images/introduction/elektrode.png}
\caption{The figure represents HD-EMG electrode that was used for recording of database used in this Thesis}
\label{fig:electrode}
\end{figure}

In addition, activation of individual motor units, i.e. individual motor neuron spike train, can be extracted from the HD-EMG recordings using Blind Source Separation methods \citep{Holobar2007, Holobar2010}, which can be a valuable information in force estimation because motor unit recruitment and firing frequency depend primarily on force level \citep{Merletti-book}. Several authors have used this approach instead of the traditional one based on intramuscular (invasive) EMG. One of the obvious advantages of this method is that is safe and not painful, although it has not been implemented in clinical practice yet. Using this technique, Holobar et al.  \citep{Holobar2010} were able to extract 6 to 7 motor units starting from contractions at 5\% MVC and up to 20\% MVC with associated discharge rates between 10 pps and 12 pps. However, one of the current limitations is that the intensity of isometric contraction must remain constant during the measurement.

HD-EMG recordings also allow calculation of two-dimensional activation maps where intensity of each pixel represents the intensity of a corresponding EMG channel (see figure \ref{fig:HD-EMG}). Consequently, the information on spatial distribution of EMG intensity over the muscle is provided. Recent studies show that changes in spatial activation pattern are related to duration of movement and fatigue \citep{Tucker2009, Staudenmann2014}, position of joint \citep{Vieira2010} and the level of contraction \citep{Holtermann2005}. 
\begin{figure}[ht]
\centering
\includegraphics[width=0.95\textwidth]{Images/introduction/HD-EMG.png}
\caption{The figure represents the HD-EMG activation map recorded on the biceps brachii muscle during flexion. Distinct activation of the two heads can be noticed in the map. Modified from \citep{Monica-thesis}}
\label{fig:HD-EMG}
\end{figure}

These HD-EMG activation maps can be also used to determine multiple innervation zones \citep{Marateb2016}, but it was also proven that this spatial characteristics of HD-EMG change depending on the  task, but also depending on the force the subject is applying, and form repeatable muscle activation pattern that can be used in identification of motion intention \citep{Rojas-Martinez2012}.

However, HD-EMG can be corrupted by low quality channels, which are a common issue in measurements due to well-known artifacts, such as: electrode displacement, bad electrical contact between skin and the electrode, movement of cables, electromagnetic interference, etc. \citep{Clancy2002b}. Affected channels differentiate themselves in amplitude and spectral content. To cope with this problem, authors in \citep{Rojas-Martinez2012} developed an expert system for detection, removal and interpolation of HD-EMG channels corrupted by artifacts. On the other hand, Ghaderi and Marateb \citep{Ghaderi2017} used image inpainting and surface reconstruction methods to reconstruct the corrupted activation map.


\section{Task identification using pattern recognition}

Given the one to one relationship between the neural commands and the activation of motor units in the muscles, surface electromyography (sEMG) has been used for more than a half of century as a noninvasive and natural way of extracting motor control information for identification of motion intention. Such information is used in numerous applications in rehabilitation engineering, e.g., prosthetics \citep{Li2010, Young2013, Stango2015}, exoskeletons \citep{VacaBenitez2013} and rehabilitation robots \citep{Dipietro2005, Marchal-Crespo2009}.

Ideally, an identification system should fulfill the following criteria \citep{Farina2014}:
\begin{itemize}
\item Intuitive control: simultaneous and proportional
\item Insensitive to changes in electrode - skin impedance,
\item Adaptive to changes during the use, i.e. fatigue, electrode-skin impedance change due to sweating and drying of conductive gel
\item Insensitive to precise position of electrodes
\item Fast and easy training procedure (ideally none)
\item Real time identification, i.e. time delay less than 300 ms \citep{Oskoei2007}
\item Low computation complexity which enables implementation in battery-powered device
\end{itemize}

In most of the commercial prosthesis \citep{Parker1986}, sEMG of two muscles is recorded. In this simple scheme a single Degree-of-Freedom (DOF) can be controlled: the EMG amplitude of one muscle controls the output of one direction, whereas the EMG amplitude of the other muscle controls the other direction. If prosthesis needs to operate in multiple DOFs, a subject needs to switch between currently active DOF either by co-contraction or by pressing a switch button. In any case, the method is not intuitive nor efficient for the user \citep{Farina2014}.

Pattern recognition is an alternative to conventional control algorithms and has been extensively used in research institutions during last decades \citep{Hakonen2015, Farina2014}. The prerequisite of using pattern recognition for task identification is the presence of a pattern that can be extracted from the EMG signal. Major advancement over conventional switching myocontrol is the possibility of instant selection of one of predefined movements.

However, although pattern recognition improves the possibilities of extraction of motion intention, it has serious limitations. Therefore, there is still a large gap between use of pattern recognition in research and in practice in rehabilitation institutes and in users' homes \citep{Jiang2012}. Pattern recognition approach does not support proportional and simultaneous control for multiple motor tasks. Therefore, consecutive tasks need to be performed sequentially. This type of control prevents the user from achieving a fluid movement, but also demands planning of movement execution. There have been solutions proposed in literature that enable simultaneous controls over multiple degrees of freedom. For example, Young et al. \citep{Young2013} propose system of parallel classifiers that use conditional probabilities to separate between combination of tasks. On the other hand, there are also publications proposing solutions for proportional control. Fougner et al. prepared a review on the topic \citep{Fougner2012}. The main idea behind this technology is that the muscle force can be estimated using the EMG signal \citep{Staudenmann2010}. 

On the other hand, one of the disadvantages of pattern recognition is the fact that in spite of the high accuracy, an error could lead to the completely unwanted task. Furthermore, although identification rate is usually very high during the stationary task, errors often occur during transition between tasks. This problems can be partially prevented by employing the e.g. majority voting principle \citep{Englehart2003}, or decision-based velocity ramp that attenuates the velocity of a movement after the change of a task \citep{Simon2011}. % Englehart 300 ms

Although crosstalk is usually considered as a negative interference in electromyography, if it is consistent and repeatable, some authors argue that it can provide a discriminative power for task identification \citep{Farina2014}. However, for some approaches it has a negative influence \citep{He2015}. To resolve this issue, source separation methods can be used to separate EMG activity of adjacent muscles \citep{Farina2004, Holobar2014}. This can be a powerful tool in task identification \citep{Naik2007}, because it could separate contributions of individual muscles in the myoelectric signal, and, therefore, minimize crosstalk effect from nearby muscles. Consequently, extracted features would characterize only the target muscles.


According to Oskoei et al. pattern-recognition-based task identification approach includes four main modules \citep{Oskoei2007}:
\begin{description}
\item[Data segmentation:] \hfill \\
Comprises various techniques and methods that are used to handle data before feature extraction. Recording must be divided in time segments on which identification will be performed. Selection of duration of time segment has effect on the identification. Features calculated on wider segments usually have lower variability and, consequently, higher repeatability and stronger pattern, which increases identification rate. On the other hand, the output of the classifier should be as fast as possible in order to be used in real time.  Therefore, the shorter the window, the shorter the response time will be. General recommendation is that the delay should be less than 300 ms \citep{Oskoei2007, Englehart2003}. 
\item[Feature extraction:] \hfill \\
This module computes and presents preselected features for a classifier. Features, instead of raw signals, are fed into a classifier to improve classification efficiency. Selection or extraction features is one of the most critical stages in myoelectric control design.
\item[Classification:] \hfill \\
A classification module recognizes signal patterns, and classifies them into predefined categories. Due to the complexity of biological signals, and the influence of physiological and physical conditions, the classifier should be adequately robust.
\item[Controller:] \hfill \\Generates output commands based on signal patterns and control schemes. Post-processing methods, such as majority voting, which are often applied after classification to eliminate destructive jumps and make a smooth output, are included in this module too.
\end{description}

Many studies agree that selection of pattern recognition technique does not have a big influence on the task identification \citep{Hakonen2015}. Therefore, simple and fast classifiers are preferred. Linear discriminant analysis has become the 'gold standard' in the field of myoelectric control because of this properties \citep{Tkach2010, Li2014, Hakonen2015}. Although this classifier assumes multivariate normal distribution of classes, experiments proved that it performs well if when normality assumption is not met \citep{Grouven1996}. Features, on the other hand, have a major influence on the identification results \citep{Englehart1999, Tkach2010}. Therefore, there are many features proposed in the literature focused on improving rate of identification of motion intention:

\begin{description}
\item[Time domain features:] \hfill \\
Mean absolute value \citep{Hudgins1993}, integrated EMG \citep{Park1998}, variance \citep{Park1998, Zardoshti1995}, root mean square \citep{Farrell2008}, waveform length \citep{Hudgins1993}, zero crossing \citep{Hudgins1993}, log detector \citep{Tkach2010}, Wilson amplitude \citep{Zardoshti1995}, slope sign change \citep{Hudgins1993}, autoregressive coefficients \citep{Hargrove2007}, cepstral coefficients \citep{Park1998}, mean absolute value slope \citep{Phinyomark2012}, histogram of EMG \citep{Phinyomark2012, Zardoshti1995}
\item[Frequency domain features:] \hfill \\
Mean frequency \citep{Phinyomark2012b}, median frequency \citep{Phinyomark2012b}, modified mean frequency \citep{Phinyomark2009}
\item[Time-frequency domain features:] \hfill \\
Short time Fourier transform \citep{Englehart2003b, Englehart2001}, continuous wavelet transform \citep{Englehart2003b, Englehart2001}, discrete wavelet transform \citep{Englehart2003b}, stationary wavelet transform \citep{Englehart2003b}, wavelet packet transform \citep{Englehart2003b, Englehart2001, Chu2006}
\item[Spatial domain features:] \hfill \\
Experimental variogram \citep{Stango2015}, center of gravity \citep{Rojas-Martinez2012, Rojas-Martinez2013}
\end{description}

Time domain features are commonly used because they achieve high identification accuracy and are computationally efficient \citep{Hakonen2015}.

Since spatial distribution contains a lot of information on the muscle, it is acknowledged as a valuable feature in identification of motion intention \citep{Stango2015, Hakonen2015, Rojas-Martinez2013}. For example, Stango et al. \citep{Stango2015} used spatial characteristics of HD-EMG recording of the forearm muscles to identify 8 hand and wrist tasks (4 degrees of freedom). They fed support vector machine classifier with a statistical measure of spatial correlation, i.e. variogram and achieved high identification results (95\% accuracy). Furthermore, they proved that proposed spatial features are robust to electrode shift.

Most of pattern recognition identification methods are subject-specific. They usually achieve very high identification results, but require time consuming training procedure for every patient individually. This could be avoided by building a single identifier for a group of patients, i.e. group-specific identifier. However, inter-subject variability is a big concern in design of a group-specific pattern recognition-based identifier. Individuals differ from each other in a lot of physiological parameters, e.g., conductivity of subcuntaneous tissue, and limb dimension. Nevertheless, by comparing HD-EMG activation maps between normal subjects it has been shown that inter-subject activation patterns exists for different tasks and levels of contraction \citep{Rojas-Martinez2012}.

In \citep{Rojas-Martinez2013} authors demonstrate that by using intensity and spatial features extracted from activation maps it is possible to construct an inter-subject identification method based on LDA classifier not only for different tasks, but also for different effort levels. Authors reported that in healthy subjects identification performance improves by adding spatial features in the identification, which proves that spatial distribution is less sensitive to inter-subject variability. They achieved sensitivity higher than 75\% for identification of four upper-limb tasks at three different effort levels and more than 90\% sensitivity when identifying only four tasks and no effort level. Also, they report higher classification results when using classification in two steps (in first step task is classified, and in the second step level of effort), rather than a single step classification.


\section{Application to patients with neuromuscular impairment}

Physical injury to the brain, spinal cord, or nerves, is usually the cause of neurological disorders. According to World Health Organization, each year there are 500 000 spinal cord injuries and 15 million strokes (of which 5 million result with death and 5 million with permanent disability) every year. Furthermore, number of people who are older than 60 years will increase to 22\% of the world population by 2050 and will count 2 billion people. Unfortunately, in affected patients motor control can be impaired as a result of damaged nerves and they often suffer from uncoordinated movements, lack of force, and spasticity. On the other hand, stroke is a serious life-threatening condition that occurs when the blood supply to the brain is interrupted, resulting in severe disability among survivors. Brain damage due to stroke can affect important areas that control everything we do, including how we move different parts of our body. Common manifestations of upper extremity motor impairment include muscle weakness, impaired motor control, and changes in muscle tone are common manifestation of upper extremity motor impairment. These impairments induce disabilities in common daily life tasks like reaching and holding objects. During recovery process, rehabilitation robots that stimulate neuroplasticity are commonly used \citep{VacaBenitez2013, Dipietro2005, Marchal-Crespo2009}. 

Patients can still have uncoordinated movements, and lack of force, or, in more difficult cases, they can weakly activate their muscles, but cannot perform the movement. If their motion intention could be extracted in real time, it would allow them to control assistive devices and maximize the benefits of robotic-aided therapies where it has been proved that the active participation improves the medical condition of the patient \citep{Hogan2006}.

It was already shown that intensity-related and task-specific activation patterns exist in patients with neurological disorders and that motion intention can be extracted from EMG. In other words, movement that patient is trying to perform can be predicted using the recorded myoelectric activity. Liu and Zhou were able to successfully perform identification of tasks using time domain and autoregressive model features in patients with incomplete spinal cord injury \citep{Liu2013}, whereas Zhang and Zhou identified tasks in patients with stroke using a similar feature set \citep{Zhang2012}.

After the neurological disorder, rehabilitation treatment should start as soon as possible, only days after injury in stroke, whereas in case of spinal cord injury, after the inflammation. Early interventions can achieve incredible results and patients can either regain control of limbs, which is known as \emph{true recovery}, or can learn new compensatory movements, which is called \emph{restitution}. 

In spite the correct neuromuscular activation, patients sometimes cannot achieve movement because of insufficient contraction force, or spasticity \citep{Liu2016b}. These patients have good chance of recovery, but therapists are often unaware of their state. On the other hand, rehabilitation robots are mostly based on force and inertia, and, therefore, cannot be of assistance either. Since this patients have the ability to generate EMG signals, they could control a rehabilitation robot and maximize their chance of recovery by individualizing rehabilitation.




\section {Doctoral thesis overview}

This Doctoral Thesis is presented as the compendium of three publications. The topic of the Thesis is the analysis of muscular patterns of upper-limb muscles during isomeric contractions and its relationship to incomplete spinal cord injury. Furthermore, method for the identification of motion intention is developed base on pattern recognition approach and muscle co-activation patterns. 

The Doctoral Thesis is organized by chapters as follows:

\begin{itemize}
\item \textbf{Chapter 2: Problem statement}\\
	This chapter states the problem and provides the objectives of Doctoral Thesis.
	
\item \textbf{Chapter 3: Spatial distribution of HD-EMG improves identification of task and force in patients with incomplete spinal cord injury}\\
	This chapter represents the first publication of the compendium of publications. Using spatial distribution of myoelectric intensity task identification was performed on patients with incomplete spinal cord injury. This work proves the positive contribution of spatial features in pattern recognition technique of identification of motor tasks. Not only that the identification rate increases, but the features show resilience to slow time dependent changes in the myoelectric signal, such as fatigue and drying of electrolytic gel
	
\item \textbf{Chapter 4: Prediction of isometric motor tasks and effort levels based on high-density EMG in patients with incomplete spinal cord injury}\\
	In this publication, the similarity of intensity and spatial distribution of intensity was investigated between patients with incomplete spinal cord injury. The results show that the repeatable pattern exists between different patients and, moreover, for the patients with similar level of injury this patterns are more similar.

\item \textbf{Chapter 5: A Novel Spatial Feature for the Identification of Motor Tasks Using High-Density Electromyography}\\
	This chapter summarizes the  third publication of the compendium. The novel feature was designed for task identification. It is based on probability density function of HD-EMG activation maps. Classifier based on this new feature show higher identification rate, as well as fidelity to fatigue.

\item \textbf{Chapter 6: Conclusions}\\
	In the last chapter, the conclusions and main contributions of the Thesis are provided. Also, the guidelines for the future work are stated, as well as list of publications derived from the Thesis.

\end{itemize}





\

