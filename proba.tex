\documentclass{article}
\usepackage{graphicx}
\usepackage{subcaption}

\begin{document}

%\chapter[Isometric and non-isometric task identification]{Identification of isometric and non-isometric upper-limb tasks using HD-EMG}
\label{ch:p4}
\textbf{Published as:} 
Jordani\'c, M., Rojas-Mart\'inez, M., Ma\~nanas, M.A., Alonso J.F., Marateb H.R.
Identification of isometric and non-isometric upper-limb tasks using HD-EMG 
\textit{REvista Iberoamericana de Autom\'atica e Inform\'atica industrial (RIAI)} Accepted for publication, 2017

doi: ------------------

Impact Factor: 0.390; Position: 57 of 60 (Q4) AUTOMATION AND CONTROL SYSTEMS; 25 OF 26 (Q4) ROBOTICS.


\textbf{Abstract:} Task identification and estimation of voluntary movement based on electromyography (EMG) is a known problem that involves different areas in expert systems, and particularly, pattern recognition with many possible applications in assistive and rehabilitation devices. The extracted information can be useful for the control of exoskeletons or robotic arms used in therapy. The emerging method of high-density electromyography opens new possibilities for extraction of neural information and it was already reported that the spatial distribution of HD-EMG intensity maps is a valuable feature in the identification of isometric tasks (contractions during which muscle fibers do not shorten). This study investigates the use of the spatial distribution of myoelectric activity in the identification of dynamic tasks performed at different velocities, similar to the tasks performed during the rehabilitation. With this objective, HD-EMG signals were recorded in a group of healthy subjects during isometric and dynamic tasks of the upper-arm. The results show that the spatial distribution is a useful feature in the identification, not only for isometric contractions, but also for dynamic contractions, improving efficiency of the control of rehabilitation devices by adapting to the user.

\textbf{Keywords:}  bioengineering; electromyography; neuromuscular; rehabilitation

\section{Introduction}
Every year there are half of a million spinal cord injuries and 15 million strokes. These injuries can have an impact on the motor control, resulting in uncoordinated movements, lack of force, and spasticity. In these cases, robot-assisted therapies can be used to stimulate neuroplasticity \citep{vanPeppen2004}. If the movement intention of the patient could be extracted in real time, this information could be used to control assistive devices in a more natural way



\end{document}