\chapter{Problem statement}
	\section{Introduction}
    
    Epilepsy is one of the most common neurological diseases, which affects 1\% of the world population \citep{Ramey2013}. The primary therapy for epileptic patients is anti-epileptic drugs that successfully eliminate seizures in about 60\% of patients \citep{Franco2014}. The most promising treatment for the remaining 40\% is surgical intervention for removing or resecting the area of the brain producing seizures \citep{Jacobs2012}, the epileptogenic zone (EZ).
    
    During the presurgical evaluation of patients, noninvasive techniques play a major role in the delimitation of the EZ. However, in some patients (25\% to 30\%) the noninvasive localization of the affected area is not possible and the use of invasive electrodes is required \citep{Pittau2014}. These electrodes are implanted in the areas where the epileptic focus is expected, providing an excellent spatial resolution in that area and a higher signal-to-noise ratio. Apart from its high invasiveness, one of the main limitations of intracranial EEG (iEEG) is that only provides a good spatial resolution in the area of implantation. Thus, iEEG does not allow the delimitation of the exact boundaries of the EZ, only providing a locally limited neurophysiological picture \citep{Muthukumaraswamy2013}. 
    
    \section{Noninvasive detection of focal epileptic activity in MEG}

The delineation of the EZ is commonly approximated by the area where the clinical seizures originate, the \emph{seizure-onset zone} (SOZ) \citep{Luders2006}. The SOZ is determined with recordings during the ictal period, but due to the unpredictability of clinical seizures, the delimitation of the SOZ is not always possible \citep{Uijl2005}. The localization of Interictal Epileptiform Discharges (IEDs) sources is a common practice in the presurgical evaluation of epilepsy \citep{Nissen2016b}, but IEDs are generated by the \emph{irritative zone} (IZ). This area partially overlaps the EZ and in some cases can appear distant to the epileptogenic lesion \citep{Tamilia2017}. The relationship between the IZ and the SOZ is still an important research issue \citep{Song2015,Melani2013,Strobbe2016}. During the last decade, high-frequency oscillations have emerged as a promising biomarker for epileptogenicity \citep{Jacobs2012}. These interictal oscillations seem to be more specific to the SOZ than IEDs \citep{Jacobs2008,Crepon2010}.

MEG is a noninvasive technique that, like scalp EEG, records the electromagnetic activity with excellent temporal resolution. Both techniques can be used to reconstruct the focal sources generating the epileptogenic activity. The  main advantage of MEG over scalp EEG is that their signals are less distorted by the high resistivity of the skull and the more conductive scalp \citep{Cuffin1979}, providing a better source localization with simpler forward models \citep{Klamer2015}. Furthermore, MEG systems are generally equipped with a larger number of sensors than common EEG systems. For these reasons MEG is more frequently used in the source space analysis. 

Noninvasive MEG is an important tool for preoperative evaluation because it can guide the placement of invasive electrodes, the current gold standard. In some cases, MEG and other noninvasive techniques can supply sufficient information for a surgical intervention without invasive recordings, thus reducing invasiveness, discomfort, and cost of the presurgical epilepsy diagnosis \citep{Aydin2015}. However, to enable the evaluation of the areas involved in epileptic processes, MEG recordings have to be clean of artifacts that distort the signal, produce inaccuracies in the localization and even might make it not possible at all. In this thesis, two types of interferences that affect MEG recordings are studied: metallic artifacts that distort the MEG signals in the frequency ranges that overlap with IEDs, and the high-frequency noise that masks the activity of HFOs. 

     \section{Objectives}
     
     	\subsection*{Main objective}
        %metodos para la reducción de ruiod
        The main objective of this thesis is the development and validation of methods for the effective noninvasive localization of interictal epileptogenic biomarkers with magnetoencephalography. 
        
        \subsection*{Specific objectives}
        
To achieve the main objective, this thesis strives for the following specific objectives:

\begin{enumerate}[I]

\item To develop algorithms, using freely-available signal analysis toolboxes, to improve the signal-to-noise ratio of MEG recordings in low and high frequencies. 
\item To compare the performance of different BSS algorithms in the reduction of metallic interference. 
\item To reduce metallic artifacts in MEG recordings automatically using blind source separation (BSS) techniques, and to evaluate the improvement of the signal-to-noise ratio quantitatively.
\item To evaluate the influence of metallic interference and its reduction in the detection and localization of interictal epileptiform discharges.
\item To reduce high-frequency noise to detect high-frequency oscillations in MEG signals.
\item To develop an automatic method to detect high-frequency oscillations in MEG signals, and to compare its detection with intracranial recordings, the current gold standard.
\item To localize the areas generating pathologic high-frequency oscillations automatically.
\item To publish the obtained results and conclusions in high-impact journals, as well as in international and national conferences.

\end{enumerate}

     \section{Thesis framework}
     
This thesis and the published articles that provide its content as a compendium were developed in the \emph{Department of Automatic Control (ESAII)} of the \emph{Universitat Polit\`{e}cnica de Catalunya (UPC)} under the framework of the brain research line of the \emph{BIOsignal Analysis for Rehabilitation and Therapy Research Group (BIOART)}, which belongs to the \emph{Biomedical Signals and Systems} division of the \emph{Biomedical Engineering Research Centre (CREB)} of UPC that belongs to the Biomedical Research Networking Center in Bioengineering, Biomaterials and Nanomedicine (CIBER-BBN). The research was done with the collaboration of the \textit{Magnetoencephalography} and the \textit{Epilepsy Units} of the \emph{Centro Médico Teknon}.	

Furthermore, this work has been supported by multiple funding projects:

\begin{enumerate}
\item Ayudas para la contratación de personal investigador novel (FI-DGR 2014). \emph{Agencia de Gestión de Ayudas Universitarias y de Investigación (AGAUR) - Generalitat de Catalunya.}

\item 	Sistemas multicanal de análisis y sensorización para rehabilitación y monitorización clínica. (DPI2011-22680) \emph{Ministerio de Economia, Industria y Competitividad (MINECO)}

\item 	Design of methods for assessing processes of neurological and neuromuscular decline associated with aging. (DPI201459049R) \emph{Ministerio de Economia, Industria y Competitividad (MINECO)}

\item	Biomedical Research Networking Center in Bioengineering, Biomaterials and Nanomedicine, CIBER-BBN, Instituto de Salud Carlos III, Spain.

\end{enumerate}
