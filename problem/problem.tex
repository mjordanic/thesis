\chapter{Problem statement}

Extraction of information on motor task intention can be used in many different application from assistive devices, prosthetics, and rehabilitation robots to leisure and gaming equipment. This information can be extracted at any point of the system for motor control: from the brain centers controlling the movement to the muscles performing the movement. The central nervous system is organized in multiple levels, from simple connections between cells to coordinated cell populations, building a complex architecture of interconnected brain regions, including the centers for motor control. All this brain activity is summed together and its electromagnetic field can be measured on the scalp surface (electroencephalography, EEG). If this information is used as an interface between the subject and the computer, it is called \emph{brain-computer-interface} (BCI). Although this approach is being extensively researched and the possibilities and achievements are rising rapidly, it is not an easy task. The problem is that activity of entire brain is superimposed to the motor control activity, such as emotions and memory.

On the other hand, EMG is electrical activity of many muscle units that carry similar information. The ratio of power of useful signal, compared to interference of other sources is much higher in EMG recordings. Moreover, by recording myoelectrical activity over muscle surface with high spatial sampling (HD-EMG), even higher SNR can be achieved and more information can be extracted. Therefore, this doctoral thesis investigates the possibilities of extraction of motor control information from multichannel sEMG during voluntary contractions.

    
    
    \section{Motivation}
    
    Voluntary movements are achieved by the contraction of skeletal muscles controlled by the Central and Peripheral Nervous system. The contraction is initiated by the release of a neurotransmitter that promotes a reaction in the walls of the muscular fiber, producing a biopotential known as Motor Unit Action Potential (MUAP) that travels from the neuromuscular junction to the tendons. The surface electromyographic signal records the continuous activation of such potentials over the surface of the skin and constitutes a valuable tool for the diagnosis, monitoring and clinical research of muscular disorders. Moreover, the use of electrode arrays facilitate the investigation of the peripheral properties of the active motor units such as: conduction velocity and fatigue \citep{Soares2015}; anatomical characteristics in terms of location of the innervation zones \citep{Beck2012}, the spatial composition of the muscle, that is, muscle compartmentalization \citep{Vieira2010}; estimation of number, type and spatial distribution of muscle fibers \citep{Marateb2016}; and change in spatial distribution of MUAPs with exercise and pain \citep{Madeleine2006}. This last property of the muscles has proven to be very useful to infer motion intention not only regarding the direction of the movement but also its force \citep{Rojas-Martinez2013}. The advantages of the HD-EMG lie in the large amount of recorded information, which enables minimizing the effect of electrodes shift and allows choosing an appropriate subset of channels for further analysis.  
%     HD-EMG also enables estimation of number, type and spatial distribution of muscle fibers \citep{Marateb2016}. This last property of the muscles has proven to be very useful to infer motion intention not only regarding the direction of the movement but also its force \citep{Rojas-Martinez2013}.
    %
%    HD-EMG enables measuring of valuable information about muscle unit recruitment: muscle fiber conduction velocity, location of the innervation zones, estimation of muscle fatigue, and estimation of number, type and the spatial distribution of muscle fibers \citep{Marateb2016}. The advantages of the HD-EMG lie in the large amount of recorded information, which enables minimizing the effect of electrodes shift and allows choosing an appropriate subset of channels for further analysis.
    
%Spinal cord injury is a neurologically disabling disease like the stroke.
In this thesis muscle co-activation patterns will be analyzed both in healthy subjects, but also in patients with incomplete spinal cord injury (iSCI). In these types of neuromuscular impairments, patients often have residual motor capabilities and can weakly activate their muscles. However, although their muscle is contracting and generating myoelectrical activity, their contraction is sometimes insufficient to generate joint movement. In this situation, it is likely that the rehabilitation will not be successful and patient could develop compensatory movements to replace the lost functionality, either by himself, or even by following therapist's advice. Moreover, rehabilitation robots are widely used in this type of rehabilitation care. However, robots most often have only force sensors and can adjust the trajectory depending on the force that patient is producing in order to assist/resist his efforts. Although it is well known that rehabilitation robots have positive effect on the therapy, their effect on the rehabilitation could be greatly improved if the robots would adjust the force and trajectory based on the patient's capabilities and efforts. Patient's effort to achieve the movement could be extracted by HD-EMG myocontrol system, which would be connected into the feedback loop of the robot to personalize the rehabilitation. It was already proven that simply moving patient's limb along a set trajectory has minimal effect on the outcome of the therapy and that therapy can be greatly improved if the patient is actively trying to achieve a movement. Therefore, personalized therapy system that responds to patient's movement intention could greatly improve the therapy.

To achieve the accurate identification of motion intention using pattern recognition, repeatable co-activation pattern should be present in the patients with neurologically disabling diseases. Therefore, the reproducibility of specific muscular activation patterns was investigated in patients with incomplete spinal cord injury during four isometric tasks of the upper limb, paying close attention to the spatial activation patterns. Moreover, activation patterns were also analyzed during different levels of effort.

The measure of pattern reproducibility can be evaluated using pattern recognition classifier in task identification. In other words, results of the task identification can be used as a figure of merit for pattern quality. If the features extracted from the HD-EMG signal form a distinct pattern for each of the tasks, and if patterns for different tasks are different, identification results will be high. Recognizable and distinct patterns will yield high identification results.
   
Task identification using pattern recognition is classification of recorded sEMG signal segments into one of predefined classes based on the characteristics of the recorded EMG signal. These extracted features should ideally form a repeatable and distinct pattern for each class, but different between classes. A variety of classifiers (e.g. hidden Markov model, support vector machine, artificial neural network, fuzzy logic and linear discriminant analysis) \citep{Oskoei2007} have been already employed in myocontrol research. Nevertheless, multiple authors agree that the identification does not depend much on the classifier type \citep{Hargrove2007, Zhang2012, Hakonen2015}. Therefore, simple and easy to train classifiers like linear discriminant analysis are preferred \citep{Li2010, Englehart1999, Tkach2010, Li2014, Hakonen2015}. On the other hand, finding an appropriate set of features is challenging \citep{Englehart1999, Tkach2010, Liu2013}.

In this work, linear discriminant analysis (LDA) and support vector machine (SVM) were used as pattern recognition classifiers. LDA is a computationally simple and efficient classifier with linear decision boundary and it is based on the Bayesian equation \citep{McLachlan2004}. It is a \emph{parametric classifier}, i.e., it estimates statistical probability of classes by estimating the probability density function of each class from the available data, which is not a simple task and can often be erroneous. On the other hand, SVM \citep{Cortes1995} is nowadays known as a very powerful classifier with a lot of different applications. The big advantage over LDA is the fact that it is a \emph{non-parametric} classifier. The model of the classifier is not obtained using assumptions of the form of the class density function and estimation of it's parameters, which is inevitably source of error. Instead, SVM forms the decision boundary using the samples (not their density estimates) by maximizing the distance between samples and the boundary. This was the idea Vladimir Vapnik, the inventor of this method stood for. It is better to try to solve the problem directly and simply, without many intermediate steps that can often be complicated and inaccurate. Detailed explanation and the working principle of these two classifiers are provided in the appendix A.

Challenges in pattern recognition in electromyography are electrode shift \citep{Hargrove2008, Young2011}, change in arm posture \citep{Fougner2011}, slow time dependent changes \citep{Farina2014}, such as fatigue \citep{Tkach2010}, and change in electrode-skin impedance \citep{Clancy2002b}. In this thesis, analysis was performed on highly controlled isometric tasks. Patients' limb was held in place using the mechanical brace, and no movement was possible. Therefore, effects accounted for limb movement, that is, electrode shift and change in arm posture were minimal. In this controlled environment, research was able to be only focused on slow time-dependent changes like fatigue. 


     \section{Objectives}
     
     	\subsection*{Main objective}

This doctoral thesis addresses the problem of extraction of information from muscular patterns obtained from multichannel surface electromyography and associated with different motor tasks. The aim of the thesis is to analyze the muscular pattern of upper-limb muscles during isomeric contractions and its relationship to neuromuscular disorders, namely to incomplete spinal cord injury. This information can be useful for the identification of motion intention, i.e. identification of intended motor task and force based on sEMG, and could provide a control signal to interfaces that control external devices, like exoskeletons or rehabilitation robots, particularly for stroke or other patients with neuromuscular disorders.




        
        \subsection*{Specific objectives}
        
To achieve the main objective, this thesis strives for the following specific objectives:

\begin{enumerate}[I]

\item To investigate muscle co-activation patterns extracted from multichannel sEMG in patients with incomplete spinal cord injury during isometric contractions, but also to evaluate the repeatability of the patterns for different motor tasks, but also for different effort levels. Patterns in intensity and in spatial domain are to be evaluated.

\item To investigate effects of fatigue on the obtained patterns

\item To investigate how this patterns change during recording time

\item To search for the similarity in multichannel sEMG activation patterns between different patients with incomplete spinal cord injury using pattern-recognition approach

\item To extract a novel spatial feature that can be used for the task identification

\item To develop a novel pattern-recognition-based procedure for identification of task and force of isometric contractions using the novel feature

\item To test the feature in task identification in healthy subjects

\item To test stability and robustness of extracted feature regarding physiological and non-physiological changes which are consequences of long-term contractions (i.e. myoelectric fatigue and gel drying).

\item To publish the obtained results and conclusions in high-impact journals and conferences.

\end{enumerate}

     \section{Thesis framework}
     
This thesis and the published articles that provide its content as a compendium were developed in the \emph{Department of Automatic Control (ESAII)} of the \emph{Universitat Polit\`{e}cnica de Catalunya (UPC)} under the framework of the brain research line of the \emph{BIOsignal Analysis for Rehabilitation and Therapy Research Group (BIOART)}, which belongs to the \emph{Biomedical Signals and Systems} division of the \emph{Biomedical Engineering Research Centre (CREB)} of UPC that belongs to the Biomedical Research Networking Center in Bioengineering, Biomaterials and Nanomedicine (CIBER-BBN). The research was done with the collaboration of the \emph{Institut Guttman in Badalona} (Spain) and the \emph{Laboratory of Engineering of Neuromuscular System and Motor Rehabilitation} at the \emph{Politecnico di Torino}.

Furthermore, this work has been supported by multiple funding projects:

\begin{enumerate}
\item Ayudas para la contratación de personal investigador novel (FI-DGR 2014). \emph{Agencia de Gestión de Ayudas Universitarias y de Investigación (AGAUR) - Generalitat de Catalunya.}

\item 	Sistemas multicanal de análisis y sensorización para rehabilitación y monitorización clínica. (DPI2011-22680) \emph{Ministerio de Economia, Industria y Competitividad (MINECO)}

\item 	Design of methods for assessing processes of neurological and neuromuscular decline associated with aging. (DPI201459049R) \emph{Ministerio de Economia, Industria y Competitividad (MINECO)}


\end{enumerate}
