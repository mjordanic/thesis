\chapter{Problem statement}
	\section{Introduction}
    
 Voluntary movements are achieved by the contraction of skeletal muscles controlled by the Central and Peripheral Nervous system. The contraction is initiated by the release of a neurotransmitter that promotes a reaction in the walls of the muscular fiber, producing a biopotential known as Motor Unit Action Potential (MUAP) that travels from the neuromuscular junction to the tendons. The surface electromyographic signal records the continuous activation of such potentials over the surface of the skin and constitutes a valuable tool for the diagnosis, monitoring and clinical research of muscular disorders. Moreover, the use of electrode arrays facilitate the investigation of the peripheral properties of the active Motor Units such as: conduction velocity and fatigue \citep{Soares2015}; anatomical characteristics in terms of location of the innervation zones \citep{Beck2012}, the spatial composition of the muscle, that is, muscle compartmentalization \citep{Vieira2010}; and change in spatial distribution of MUAPs with exercise and pain \citep{Madeleine2006}. This last property of the muscles has proven to be very useful to infer motion intention not only regarding the direction of the movement but also its power \citep{Rojas-Martinez2013}.
    
    \section{Task identification}

Something on task identification Problems 

     \section{Objectives}
     
     	\subsection*{Main objective}
        %metodos para la reducción de ruiod
This doctoral thesis addresses the problem of extraction of information from muscular patterns obtained from multichannel surface electromyography and associated with different movement directions. The aim of the thesis is to analyze the muscular pattern of upper-limb muscles during isomeric contractions and its relationship to neuromuscular disorders, particularly to incomplete spinal cord injury. This information can be useful for the identification of motion intention, i.e. identification of intended motor task and force based on EMG.

        
        \subsection*{Specific objectives}
        
To achieve the main objective, this thesis strives for the following specific objectives:

\begin{enumerate}[I]

\item To develop a pattern recognition-based procedure for identification of task and force of isometric contractions (i.e. extract motor intention). For this assignment, different features and methods for their selection as well as classification techniques will be evaluated and compared in order to choose the best-suited solution. Special attention will be paid to features related to spatial distribution of myoelectric intensity recorded over the surface of the muscle. This is a new and unexplored approach that has proven to have high potential in identification. 

\item To test stability and robustness of extracted features regarding physiological and non-physiological changes which are consequences of long-term contractions (i.e. myoelectric fatigue and gel drying).

\item To publish the obtained results and conclusions in high-impact journals, as well as in international and national conferences.

Methods  were applied to control subjects as well as to patients with incomplete spinal cord injury with reduced mobility of the upper-limb.

\end{enumerate}

     \section{Thesis framework}
     
This thesis and the published articles that provide its content as a compendium were developed in the \emph{Department of Automatic Control (ESAII)} of the \emph{Universitat Polit\`{e}cnica de Catalunya (UPC)} under the framework of the brain research line of the \emph{BIOsignal Analysis for Rehabilitation and Therapy Research Group (BIOART)}, which belongs to the \emph{Biomedical Signals and Systems} division of the \emph{Biomedical Engineering Research Centre (CREB)} of UPC that belongs to the Biomedical Research Networking Center in Bioengineering, Biomaterials and Nanomedicine (CIBER-BBN). The research was done with the collaboration of the Institut Guttman in Badalona (Spain) and the Laboratory of Engineering of Neuromuscular System and Motor Rehabilitation at the Politecnico di Torino.

Furthermore, this work has been supported by multiple funding projects:

\begin{enumerate}
\item Ayudas para la contratación de personal investigador novel (FI-DGR 2014). \emph{Agencia de Gestión de Ayudas Universitarias y de Investigación (AGAUR) - Generalitat de Catalunya.}

\item 	Sistemas multicanal de análisis y sensorización para rehabilitación y monitorización clínica. (DPI2011-22680) \emph{Ministerio de Economia, Industria y Competitividad (MINECO)}

\item 	Design of methods for assessing processes of neurological and neuromuscular decline associated with aging. (DPI201459049R) \emph{Ministerio de Economia, Industria y Competitividad (MINECO)}


\end{enumerate}
